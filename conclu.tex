\chapter{Conclusions and perspectives}

In this thesis, we studied the swimming of bacteria \textit{Escherichia coli} (\textit{E. coli}) near a sinusoidal boundary. We used low-density suspensions of bacteria to measure their accumulation near the curved surfaces. Through image analysis and bacterial tracking, we measured bacteria's density, speed, and residence times in the region close to the curved wall. We also developed a model of bacteria swimming, considering steric interactions with the surface.

We observed that bacteria movement in the curved surface depends on the parameters of the sinusoidal shape, namely the amplitude $A$ and the wavelength $\lambda$. Their combined effect can be largely summarized in the wall maximum curvature $\kappa=4\pi^2 A /  \lambda^2$. For lower curvatures, bacteria move along the surface quickly, and when bacteria reach a peak, they may leave the surface. As the curvature increases, bacteria become trapped in the valleys and eventually form clusters that last for several seconds. We called this transition the accumulation transition. To characterize this transition, we measured the mean density of bacteria in a period of the sinusoidal wall, through the intensity of the fluorescence of bacteria. We focused on the intensity in the curved wall, which we normalized by the mean intensity in the flat wall. This normalization ensures experiments are comparable. The accumulation transition was characterized via the Fourier coefficient $c_1$ associated with the shape of the intensity profile. We determined that the accumulation transition occurs near a critical curvature $\kappa^* =$ \SI{0.3}{\per \micro \meter}. 

The model considered interactions between bacteria and the wall through the steric alignment intensity $K$ and random reorientations induced by the liquid particles through the rotational diffusion coefficient $D_r$. $K$ and $D_r$ were the only non-fixed parameters of the model. We adjusted the model parameters to replicate the values of $c_1$ resulting in the optimal values $K^*=$ \SI[per-mode = symbol]{3.0}{\radian\per\second} and $D_r^*=$ \SI[per-mode = symbol]{0.015}{\square\radian\per\second}. This values compare well to the reported measurements in the literature. In a flat surface $K=$ \SI[per-mode = symbol]{4.9}{\radian\per\second} was measured \cite{Bianchi20193DInterface}, and a lower value of $K$ is to be expected as bacteria align less with the curved surface. Also $D_r=$ \SI[per-mode = symbol]{0.057}{\square\radian\per\second} was reported for cells swimming far away from boundaries \cite{Drescher2011FluidScattering}. In our experiments, bacteria in the focal swim in contact with a frontal surface, so it is reasonable to have a lower value of $D_r$ in our case due to the geometric restrictions that the surfaces imposes on the liquid and the rotation of the bacteria. 

The optimal values of the model parameters were determined only with $c_1$. Nevertheless, the model replicates other quantities such as mean accumulation in the curved wall, speed profiles near the wall, and contact times with it. This is interesting because the model is very simple. The model does not consider the spherocylindrical shape of cells, the friction with the wall, the collision and alignment between cells, and the hydrodynamic effects caused by the flagella movement. There is plenty of physics involved in these experiments, but a model with minimal ingredients predicts the observed accumulation transition and makes reasonable predictions for all the measurements we made in the experiments. This can mean only one thing, the dynamics of cells near sinusoidal walls is dominated by the effects described in the model. We conclude that the steric alignment of cells with the wall and the rotational diffusion are the primary physical mechanisms that govern the dynamics of bacteria when swimming near a curved surface.

Regarding the mean accumulation of bacteria, we discovered that near the critical curvature, a minimum of the average accumulation along the curved wall is found for the values $A=$ \SI{5.6}{\micro\meter}, $\lambda= $ \SI{27}{\micro\meter}. Nevertheless, an experiment with almost equal curvature but lower amplitude showed a higher accumulation. This is because, for these curvatures, bacteria move around the valley easily, but for higher amplitude, the bacteria leave the wall with a higher inclination and therefore move away from the surface. This is interesting for designing surfaces that reduce biofilm formation because it reveals that curvature is not the only relevant parameter. We believe that semicircular patterns are the best option to reduce accumulation of bacteria, as they are expelled with the highest inclination possible. Nevertheless, our characterization as a function of the curvature is helpful because it is independent of the shape. If we extrapolate the results of this thesis for the semicircular geometery, we can predict an optimal radius $R^*$ around $R^*=(\kappa^*)^{-1}\appox$ \SI{5}{\micro\meter} for semicircular patterns.

Thanks to the measurments of residence time of bacteria near the walls, we determined that the average contact time in the sinusoidal wall is an order of magnitude lower than in the flat surfaces. This decrease could mean that bacteria do not have time to adhere to the surface. In the future, we believe it is necessary to test this surface in more natural situations to measure its effectiveness in preventing biofilm formation. 

The work done in this thesis still has plenty of room for improvement, as was discussed in the previous chapter. For example, we only have experiments with $A\approx$ \SI{9}{\micro\meter} in one day. Due to technical problems with the camera, it was impossible to generate more experimental data. We are also lacking resolution in the amplitudes, because for the wavelengths \SIlist{21;24}{\micro\meter} we cannot observe the accumulation transition optimally. Increasing the number of experiments in this range of amplitudes is crucial to elucidate the differences observed in this range. Therefore an increase of the resolution in the  $(\lambda, \ A)$ space will better describe the accumulation transition. Moreover, the experiments will benefit from data from lower amplitudes as it is expected that for curvatures near zero the behavior in the curved wall will recover the behavior of the flat surface. We also believe that a tracking-based bacterial density measure will allow a more direct comparison between experiments and simulations, but it was not considered due to time restrictions.

In summary, this thesis contributes to understanding how shape alters the accumulation of bacteria near a surface. We observe a transition in the accumulation depending on the curvature in the wall. We explain this phenomenon by a simple model that considers an alignment with the wall and rotational diffusion. By adjusting the model parameters to replicate the transition, the dependence of bacterial speed and mean density with respect to the curvature is replicated, indicating that the system's dynamics are correctly represented. This implies that the steric alignment dominates other effects typically present in experiments but not included in the model. In addition, we have shown that these surfaces reduce the accumulation of bacteria on the wall considerably as the bacteria are in contact for times an order of magnitude shorter than near flat walls. This study promises that control of biofilm formation by optimization of surface curvature is possible.
