\begin{adjustwidth}{4.2cm}{}
\begin{tabular}{l}
	RESUMEN DE LA MEMORIA PARA OPTAR AL GRADO\\
	DE MAGÍSTER EN CIENCIAS, MENCIÓN FÍSICA \\
	POR: BENJAMÍN PÉREZ \\
	FECHA: \MakeUppercase{\today} \\
	PROF. GUÍA: RODRIGO SOTO, MARÍA LUISA CORDERO, \\
	NÉSTOR SEPÚLVEDA\\
\end{tabular}
\end{adjustwidth}

\begin{center}
    \MakeUppercase{Caracterización del nado de \textit{E.coli} cerca de paredes sinusoidales }
\end{center}

Las bacterias nadan gracias al movimiento de sus flagelos. Esa afirmación se ramifica de muchas maneras, ya que las bacterias tienen diferentes formas corporales y tipos de flagelos. Además, el entorno tiene un papel crucial. No es lo mismo nadar en el océano con flujos o cerca de una superficie. Se ha visto que las superficies planas atrapan a las bacterias, eventualmente provocando su adhesión a la superficie y el inicio de la formación de biofilms. Evitar la formación de biofilm es un problema abierto cuya solución salvaría vidas. En esta tesis se estudiará experimental y teóricamente, cómo la forma de la superficie puede modificar el atrapamiento de las celulas. La idea principal es que una pared microscópica sinusoidal podría reorientar las células, expulsandolas lejos de la pared. 

El primer capítulo, explica los conceptos principales requeridos para entender este trabajo y su relevancia. El segundo capítulo describe los protocolos de cultivo de bacterias, la fabricación de los dispositivos microfluídicos, las mediciones y su análisis. Los experimentos se hicieron con una cepa modificada genéticamente de \textit{E.coli} que no hace giros porque eso supone una menor probabilidad de abandonar una superficie. La densidad se mantuvo baja, para poder observar el movimiento de bacterias individuales.

El tercer capítulo presenta un marco teórico que describe como simular numéricamente la dinámica de las bacterias con ingredientes mínimos. Esto nos lleva a un modelo microscópico de partículas brownianas activas esféricas en una representación bidimensional que considera colisiones elásticas y alineamientos estéricos con la pared.

El capítulo 4 muestra los resultados obtenidos en los experimentos y el modelo, que demuestran que la curvatura de la pared sinusoidal juega un papel fundamental. Cuando la pared curva es casi plana, las bacterias apenas salen de la pared. Por otro lado, si el valle es demasiado estrecho, las bacterias quedarán atrapadas ahí. Variando la amplitud y la longitud de onda del perfil de la superficie, se encuentra una transición entre estos dos regímenes. El punto crítico representa el caso en que las bacterias aún pueden moverse por el valle, pero el ángulo de escape es mayor provocando que las bacterias salgan de la superficie, causando un mínimo en la acumulación. Mediciones de la velocidad vía tracking apoyan este resultado. El modelo numérico reproduce cualitativamente las observaciones experimentales ajustando solo dos parámetros, el coeficiente de difusión rotacional y la magnitud del alineamiento con la pared. Esta simplicidad permite concluir que el alineamiento de las células con la pared es la causa de este fenómeno, mientras que otros efectos causados por interacciones hidrodinámicas son despreciables. Debido a que muchas bacterias experimentan fuerzas estéricas con la pared de forma similar, este estudio promete aplicar a otras especies de bacterias.

El capítulo 5 resume las conclusiones y perspectivas de este trabajo.

