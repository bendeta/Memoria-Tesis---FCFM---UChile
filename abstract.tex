\begin{adjustwidth}{4.2cm}{}
\begin{tabular}{l}
	\MakeUppercase{abstract of the thesis for the degree } \\
	\MakeUppercase{of master of science, mention in physics}\\
	BY: BENJAMÍN PÉREZ \\
	DATE: \MakeUppercase{\today} \\
	SUPERVISORS: MARÍA LUISA CORDERO, NÉSTOR SEPÚLVEDA, \\ 
    RODRIGO SOTO\\
\end{tabular}
\end{adjustwidth}

\begin{center}
    \MakeUppercase{Characterization of \textit{E.coli} swimming near sinusoidal surfaces }
\end{center}

Bacteria swim thanks to the movement of their flagella. That affirmation branches in many forms, as bacteria have different body shapes and flagella types. Moreover, the environment plays a crucial role. Swimming in the ocean with flows or near a surface is not the same. Flat surfaces have been found to trap bacteria, eventually resulting in their adhesion to the surface and the initiation of biofilm formation. Avoiding biofilm formation is an open medical problem whose solution would save lives. This thesis studies, both experimentally and theoretically, how surface shape can modify cell trapping. The main idea is that a microscopic sinusoidal wall could reorient cells and expel them away from the wall. 

The first chapter explains the main concepts required to understand this work and its relevance. The second chapter describes the protocols for bacteria culture, fabrication of the microfluidic devices, data acquisition, and analysis. Experiments were performed with a genetically modified strain of \textit{E.coli} that does not tumble because it is less likely for them to leave the surface. Also, bacterial density was kept low to observe individual bacteria movement. 

The third chapter presents a theoretical framework for the numerical description of the bacterial dynamics with minimal components. This leads us to an agent-based model of spherical active Brownian particles in a two-dimensional representation that considers elastic collisions and steric alignments with the wall.

Chapter 4 shows the results obtained in the experiments and the model, which show that the curvature of the sinusoidal wall plays a fundamental role. When the curved wall is almost flat, the bacteria hardly come out of the wall. On the other hand, if the valleys are too narrow, bacteria will be trapped there. Varying the amplitude and wavelength of the surface profile, a transition between these two regimes is found. The critical regime represents the case where bacteria can still move through the valley quickly, but the escape angle is higher, causing the bacteria to leave the surface, leading to a minimum in the accumulation. Measured velocities in the tracking of bacteria support this result. The numerical model qualitatively reproduces experimental observations adjusting only two parameters, the rotational diffusion coefficient and the magnitude of the alignment interaction with the wall. This simplicity allows us to conclude that the alignment of the cells with the wall is the cause of this phenomenon, while other effects caused by hydrodynamic interactions with the wall and between cells are negligible. Because many bacteria experience steric forces in a similar way, this study promises to apply to other bacterial species. 

Finally, chapter 5 summarizes the conclusions and perspectives of this work. 


\newpage