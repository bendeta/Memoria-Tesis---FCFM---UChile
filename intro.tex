\chapter{Introducción}

Biofilm is a structure difficult to deal with, as antibiotics have proven to fail at killing bacteria in biofilm even at a concentration 1000 times higher than the usual concentration that kills floating bacteria \cite{introduction to biofilm}. This means that biofilm is a chronical bacterial infection.

Biofilm was not a problem in the early development of health care, because is fairly rare to have this kind of infections inside your body. Due to this, biofilm was the last of the problems but this has changed recently. The reason for this change relies on the cause of biofilm formation. "The inability of the host and of therapeutic efforts to resolve acute infection triggers a series of events that culminates in a chronic condition" \cite{treatment of chronic infection}. Diabetes and intra-corporal devices are contemporary reasons that decrease the ability of the body to resolve an infection so biofilm appears. Badly treated diabetes can produce chronic hyperglycemia, associated with failure of blood vessels among many other organs \cite{diagnosis and classification of diabetes mellitus}. This reduces wound healing as less blood reaches the wound, therefore people with diabetes are an at-risk group for biofilm formation. On the other hand intra-corporal devices are made to replace something on your body, compensating for the missing function. The problem is that for the inmune system, intra-corporal devices are like dead tissue, without any blood. This means that any bacteria that attach to this surface will be harder to reach. This device-related infections have interrupted the development of complex medical devices, that could replace organs like the heart. If these mechanical organs are susceptible to infections, they bring more problems than solutions \cite{introduction to biofilm}.

Considering this, developing technology that prevents biofilm formation has brought interest as it could allow 