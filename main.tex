% se puede agregar la opción [english] para 
%  memorias o tesis en inglés (borrando el archivo .aux)
\documentclass{umemoria} 

\depto{Departamento de Física}
\author{Benjamín Ignacio Pérez Estay}
\title{Characterization of \textit{E.coli} swimming near sinusoidal surfaces }

% incluir ambos comandos para una doble titulación
%  o quitar el comando que no aplica
\tesis{Magíster en Ciencias, mención Física}
%\tesis{Doctor en ???} % incluir solo este comando para doctorados

% puede haber varios profesores guía seperados por coma;
% pero si es una memoria, solo puede haber un profesor guía
\guia{María Luisa Cordero, Néstor Sepúlveda, Rodrigo Soto} 

% puede haber varios profesores co-guía seperados por coma;
% pero si es una memoria, el profesor co-guía será el primer
% integrante de la comisión
%\coguia{Nombre Completo Co-Guía} % incluir en caso de co-guía de *tesis*

%\cotutela{Nombre Institución} % incluir en caso de cotutela

\comision{Claudia Trejo, Sergio Rica}

%\auspicio{Nombre Institución} % incluir en caso de recibir financiamiento

% tiene que ser el año en que se da el examen de título/grado (defensa)
%\anho{2021} % incluir solo para reemplazar el año actual

\usepackage{lipsum}
\usepackage{xcolor}
\usepackage{svg}
\usepackage{float}
\usepackage{wrapfig, blindtext}
\usepackage{siunitx}
\usepackage{makecell}
\usepackage{tabularx}
\usepackage{afterpage}
\usepackage{fancyhdr}
\usepackage{changepage}
\usepackage{cancel}
\DeclareSIUnit\rpm{rpm}
\DeclareSIUnit\molar{M}
\DeclareSIUnit\OD{OD$_{600}$}
\DeclareSIUnit\cells{cells}
\DeclareSIUnit\pixels{px}
\bibliographystyle{ieeetr}

\setlength{\parindent}{0pt}

\begin{document}

\sloppy
\frontmatter
\maketitle

\begin{adjustwidth}{4.2cm}{}
\begin{tabular}{l}
	\MakeUppercase{abstract of the thesis for the degree } \\
	\MakeUppercase{of master of science, mention in physics}\\
	BY: BENJAMÍN PÉREZ \\
	DATE: \MakeUppercase{\today} \\
	SUPERVISORS: MARÍA LUISA CORDERO, NÉSTOR SEPÚLVEDA, \\ 
    RODRIGO SOTO\\
\end{tabular}
\end{adjustwidth}

\begin{center}
    \MakeUppercase{Characterization of \textit{E.coli} swimming near sinusoidal surfaces }
\end{center}

Bacteria swim thanks to the movement of their flagella. That affirmation branches in many forms, as bacteria have different body shapes and flagella types. Moreover, the environment plays a crucial role. Swimming in the ocean with flows or near a surface is not the same. Flat surfaces have been found to trap bacteria, eventually resulting in their adhesion to the surface and the initiation of biofilm formation. Avoiding biofilm formation is an open medical problem whose solution would save lives. This thesis studies, both experimentally and theoretically, how surface shape can modify cell trapping. The main idea is that a microscopic sinusoidal wall could reorient cells and expel them away from the wall. 

The first chapter explains the main concepts required to understand this work and its relevance. The second chapter describes the protocols for bacteria culture, fabrication of the microfluidic devices, data acquisition, and analysis. Experiments were performed with a genetically modified strain of \textit{E.coli} that does not tumble because it is less likely for them to leave the surface. Also, bacterial density was kept low to observe individual bacteria movement. 

The third chapter presents a theoretical framework for the numerical description of the bacterial dynamics with minimal components. This leads us to an agent-based model of spherical active Brownian particles in a two-dimensional representation that considers elastic collisions and steric alignments with the wall.

Chapter 4 shows the results obtained in the experiments and the model, which show that the curvature of the sinusoidal wall plays a fundamental role. When the curved wall is almost flat, the bacteria hardly come out of the wall. On the other hand, if the valleys are too narrow, bacteria will be trapped there. Varying the amplitude and wavelength of the surface profile, a transition between these two regimes is found. The critical regime represents the case where bacteria can still move through the valley quickly, but the escape angle is higher, causing the bacteria to leave the surface, leading to a minimum in the accumulation. Measured velocities in the tracking of bacteria support this result. The numerical model qualitatively reproduces experimental observations adjusting only two parameters, the rotational diffusion coefficient and the magnitude of the alignment interaction with the wall. This simplicity allows us to conclude that the alignment of the cells with the wall is the cause of this phenomenon, while other effects caused by hydrodynamic interactions with the wall and between cells are negligible. Because many bacteria experience steric forces in a similar way, this study promises to apply to other bacterial species. 

Finally, chapter 5 summarizes the conclusions and perspectives of this work. 


\newpage
\begin{adjustwidth}{4.2cm}{}
\begin{tabular}{l}
	RESUMEN DE LA MEMORIA PARA OPTAR AL GRADO\\
	DE MAGÍSTER EN CIENCIAS, MENCIÓN FÍSICA \\
	POR: BENJAMÍN PÉREZ \\
	FECHA: \MakeUppercase{\today} \\
	PROF. GUÍA: RODRIGO SOTO, MARÍA LUISA CORDERO, \\
	NÉSTOR SEPÚLVEDA\\
\end{tabular}
\end{adjustwidth}

\begin{center}
    \MakeUppercase{Caracterización del nado de \textit{E.coli} cerca de paredes sinusoidales }
\end{center}

Las bacterias nadan gracias al movimiento de sus flagelos. Esa afirmación se ramifica de muchas maneras, ya que las bacterias tienen diferentes formas corporales y tipos de flagelos. Además, el entorno tiene un papel crucial. No es lo mismo nadar en el océano con flujos o cerca de una superficie. Se ha visto que las superficies planas atrapan a las bacterias, eventualmente provocando su adhesión a la superficie y el inicio de la formación de biofilms. Evitar la formación de biofilm es un problema abierto cuya solución salvaría vidas. En esta tesis se estudiará experimental y teóricamente, cómo la forma de la superficie puede modificar el atrapamiento de las celulas. La idea principal es que una pared microscópica sinusoidal podría reorientar las células, expulsandolas lejos de la pared. 

El primer capítulo, explica los conceptos principales requeridos para entender este trabajo y su relevancia. El segundo capítulo describe los protocolos de cultivo de bacterias, la fabricación de los dispositivos microfluídicos, las mediciones y su análisis. Los experimentos se hicieron con una cepa modificada genéticamente de \textit{E.coli} que no hace giros porque eso supone una menor probabilidad de abandonar una superficie. La densidad se mantuvo baja, para poder observar el movimiento de bacterias individuales.

El tercer capítulo presenta un marco teórico que describe como simular numéricamente la dinámica de las bacterias con ingredientes mínimos. Esto nos lleva a un modelo microscópico de partículas brownianas activas esféricas en una representación bidimensional que considera colisiones elásticas y alineamientos estéricos con la pared.

El capítulo 4 muestra los resultados obtenidos en los experimentos y el modelo, que demuestran que la curvatura de la pared sinusoidal juega un papel fundamental. Cuando la pared curva es casi plana, las bacterias apenas salen de la pared. Por otro lado, si el valle es demasiado estrecho, las bacterias quedarán atrapadas ahí. Variando la amplitud y la longitud de onda del perfil de la superficie, se encuentra una transición entre estos dos regímenes. El punto crítico representa el caso en que las bacterias aún pueden moverse por el valle, pero el ángulo de escape es mayor provocando que las bacterias salgan de la superficie, causando un mínimo en la acumulación. Mediciones de la velocidad vía tracking apoyan este resultado. El modelo numérico reproduce cualitativamente las observaciones experimentales ajustando solo dos parámetros, el coeficiente de difusión rotacional y la magnitud del alineamiento con la pared. Esta simplicidad permite concluir que el alineamiento de las células con la pared es la causa de este fenómeno, mientras que otros efectos causados por interacciones hidrodinámicas son despreciables. Debido a que muchas bacterias experimentan fuerzas estéricas con la pared de forma similar, este estudio promete aplicar a otras especies de bacterias.

El capítulo 5 resume las conclusiones y perspectivas de este trabajo.



\begin{dedicatoria}
Por todas las bacterias que vivieron y murieron en esta tesis
\end{dedicatoria}

\begin{thanks}
Esta tesis, ha sido el fruto de mucho trabajo a lo largo de los últimos dos años. Estos han sido años complicados debido a la pandemia mundial. Pero es en los momentos díficiles cuando apreciamos a las personas que nos rodean. Es por esto que quiero agradecer a todos quienes me han apoyado.

Primero, quiero agradecer a mi polola y mejor amiga Pauli. Vivir juntos ha sido increíble. Cocinar, ver series, hablar y reírme contigo me hace muy feliz. Contar con alguien que apañe en los momentos de trabajo y de descanso es muy importante para mí. Estoy emocionado por empezar esta nueva etapa contigo en Europa, que se que nos divertiremos mucho. 

Para mis excelentes profesores, Rodrigo, María Luisa y Néstor, gracias por todas las ideas que han compartido conmigo. Aprendí de ustedes a hacer modelos y experimentos, dandome una perspectiva mucho más profunda de mi trabajo, lo cual valoro mucho. Ustedes saben mejor que nadie la dedicación que le hemos puesto a esta investigación y sin su ayuda no sería lo mismo. Me alegro de saber que sus enseñanzas me acompañaran en mi vida como cientifico. Han hecho un trabajo excelente.

A mis colegas, Kevin y Cristian, les agradezco por enseñarme a hacer experimentos con bacterias. Hace un año, para mí las bacterias eran un objeto en mis simulaciones, pero gracias a ustedes ahora me encuentro pensando en nuevos experimentos para explorar en la frontera del conocimiento. Además, les agradezco su ayuda y compañía en los días de laboratorio. En parte por ustedes es que he disfrutado tanto haciendo experimentos.

También quiero agradecer a mis amigos, Luquitas, Mau, Byron, Paula, Pauli, Robert, Raza y Nico, que me acompañaron durante la carrera, gracias por hacerme disfrutar de todos esos momentos junto a ustedes. Realmente hicieron que estudiar hasta tarde fuera divertido. Las conversaciones interesantes, los juegos de mesa, las oncecitas y el estudio con ustedes son mis recuerdos más preciado en la universidad. También a mis amigos Talo, Alonso, Diegui y Piña, gracias por todas esas horas de juegos junto a ustedes, que tanto se necesitan cuando hay que descansar.

Por último, agradezco a mi familia, Pía, Sergio, Cata y Nico, por su apoyo incondicional en todos los momentos de mi vida. Somos una familia muy unida y eso es muy importante para mí. Quiero que sepan que aunque ahora me vaya fuera del país, siempre estaremos juntos y que los quiero mucho.

\end{thanks}

\tableofcontents
\listoftables % opcional
\listoffigures % opcional

\mainmatter

\chapter{Introducción}

Biofilm is a structure difficult to deal with, as antibiotics have proven to fail at killing bacteria in biofilm even at a concentration 1000 times higher than the usual concentration that kills floating bacteria \cite{introduction to biofilm}. This means that biofilm is a chronical bacterial infection.

Biofilm was not a problem in the early development of health care, because is fairly rare to have this kind of infections inside your body. Due to this, biofilm was the last of the problems but this has changed recently. The reason for this change relies on the cause of biofilm formation. "The inability of the host and of therapeutic efforts to resolve acute infection triggers a series of events that culminates in a chronic condition" \cite{treatment of chronic infection}. Diabetes and intra-corporal devices are contemporary reasons that decrease the ability of the body to resolve an infection so biofilm appears. Badly treated diabetes can produce chronic hyperglycemia, associated with failure of blood vessels among many other organs \cite{diagnosis and classification of diabetes mellitus}. This reduces wound healing as less blood reaches the wound, therefore people with diabetes are an at-risk group for biofilm formation. On the other hand intra-corporal devices are made to replace something on your body, compensating for the missing function. The problem is that for the inmune system, intra-corporal devices are like dead tissue, without any blood. This means that any bacteria that attach to this surface will be harder to reach. This device-related infections have interrupted the development of complex medical devices, that could replace organs like the heart. If these mechanical organs are susceptible to infections, they bring more problems than solutions \cite{introduction to biofilm}.

Considering this, developing technology that prevents biofilm formation has brought interest as it could allow 
\chapter{Experiments}

\section{Experimental Protocols}
\subsection{Bacteria culture}

Experiments were done with the non-chemotactic, smooth swimmer strain of \textit{E.coli} JEK1038 (W3110 [lacZY::GFPmut2, cheY::frt], green) provided by prof. Juan Keymer. The strain was modified to express the green fluorescent protein GFPmut2, and its run-and-tumble dynamics were suppressed by cheY deletion \cite{VanVliet2014ThePopulations}. 

We put \SI{20}{\micro\liter} of bacteria stock at \SI{-20}{\degreeCelsius} (appendix?) in \SI{5}{\milli\liter} of Lysogeny Broth (LB) medium for approximately 24 hours in an incubator with a shaker at \SI{28}{\degreeCelsius} and \SI{180}{\rpm}. Then, \SI{30}{\micro\liter} of this overnight were diluted in \SI{3}{\milli\liter} of LB medium with \SI{3}{\milli\molar} isopropyl $\beta$-D-1-thiogalactopyranoside (IPTG Sigma-Aldrich) and grown until the optical density at \SI{600}{\nano\meter} (\SI{}{\OD}) reaches $0.5\pm0.05$. Afterward, we added $0.1\%$ bovine serum albumin to avoid cell-to-cell adhesion and centrifuge the culture for $15$ minutes at \SI{4600}{\rpm} or $2600$ relative centrifugal force (rcf), leaving a bacteria pellet at the bottom of the falcon tube. We resuspended the pellet in \SI{3}{\milli\liter} of MMA (appendix?), resulting in a mixture with an \SI{}{\OD} slightly lower than $0.5$. To reach a low-density, we again diluted until \SI{5d-4}{\OD} is reached or approximately \SI[per-mode = symbol]{4d6}{\cells\per\milli\liter}. It is important to mention that \SI{}{\OD} is insufficient to determine final density at our experiments since bacteria will not enter the channel evenly every time because they move through the walls. We assume that these density variations are sufficiently small not to affect the dynamics of each regime.

\subsection{Fabrication of microfluidic devices}


We fabricated the microfluidic devices used in the experiments with conventional optical lithography techniques. \textcolor{red}{Here will write lines about how we do optical lithography}. The mold is then put in a petri dish and filled with PDMS. 

\afterpage{%
\begin{figure}[H]
	\centering
	\includesvg[scale=1]{imagenes/channel diagrams}
	\caption[Microfluidic device diagram]{a) Diagram of the 3D perspective of the channel for different dimensions, not scaled. The frontal side of this sketch will face towards the microscope slide after forming the PDMS-PDMS bond. The curved wall has a sinusoidal form with one amplitude $A$ and different wavelengths $\lambda$. There are four different channels with amplitudes $A=$ \SIlist[list-units=single, list-final-separator = {, }]{3;6;9;12}{\micro\meter} and each is divided into four sections with wavelengths $\lambda=$ \SIlist[list-units=single, list-final-separator = {, }]{21;24;27;30}{\micro\meter} allowing to study in a range of curvatures.  b) diagram of the empty microfluidic device. The red line is the channel where all the experiments are performed. c) diagram of the microfluidic device when bacteria suspension is added. The bacteria suspension, represented as green, fulfills both pools and the channel.}
	\label{channel_diagram}
\end{figure}
}

We prepare a PDMS mixture of Sylgard 184 elastomer base and curing agent in a 10:1 mass ratio. It is essential to mix for many seconds to ensure the PDMS is homogenous. If the mixing is not enough, some parts of the PDMS might not separate from the mold and cause irregularities on the channel. Then, the mixture should be centrifuged for 10 minutes at 5000 rpm to degas it. To ensure there is no air after pouring the mixture on the mold, it is necessary to put the mold in a vacuum chamber. The remaining bubbles will expand and merge, so they pop more easily. Finally, the mold is left in an oven at \SI{65}{\degreeCelsius} for at least $1$ hour. If the air bubbles were not removed, they expand during the heating process and could ruin the shape of the channel. After removing each channel from the mold, we made two entrance pools with a \SI{4}{\milli\meter} tissue punch for each channel.

We also considered that the channel must have all four walls of PDMS to avoid different mechanical or chemical properties on the walls. To do so, we cover a microscope slide with a thin layer of $\sim0.4g$ of PDMS, spread with a plastic spatula. The slide is left overnight on top of a leveled surface, so PDMS uniformly distributes, then put into the oven. Using a plasma cleaner, it is possible to bond the channel with the slide \cite{Henry2015ScholarlyCommonsProtocol-Technics}. We set the RF level to max power and exposed the PDMS block with the microchannel and the PDMS-covered glass to air plasma for 1 minute. PDMS is comprised of repeated units of -O-Si(CH$_3$)$_2$. The exposure to an oxygen plasma will form silanol groups Si-OH, so when a similar surface is brought into contact, the covalent Si-O-Si bonds are created, displacing a water molecule \cite{Koh2012QuantitativeEffect}. Finally, plasma oxidation will make the channel surface hydrophilic. The final assembly is shown in the figure \ref{channel_diagram} b).

The channels are \SI{100}{\micro\meter} height and \SI{25}{\micro\meter} deep and have three flat walls and one curved in a sinusoidal form. The channel is divided in several sections with different combinations of amplitude $A$ and wavelength $\lambda$. Figure \ref{channel_diagram} a) shows a diagram of the channel. The nominal values of the amplitudes are  $A=$ \SIlist[list-units=single, list-final-separator = {, }]{3;6;9;12}{\micro\meter} and the wavelengths $\lambda=$ \SIlist[list-units=single, list-final-separator = {, }]{21;24;27;30}{\micro\meter}. The real values of these dimensions were measured for every channel section, differing at most on \SI{1}{\micro\meter} from their nominal values. The measured dimensions will be used to characterize channels. 


\subsection{Experimental setup}

\begin{wrapfigure}{r}{0.5\linewidth}
\centering
\includesvg[width=\linewidth]{imagenes/focal plane}
\caption[Focal plane diagram]{Diagram of the focal plane. The width of the focal plane is \SI{2}{\micro\meter} so the depth of the channel does not matter in the observed dynamics. }
\label{focal_plane}
\end{wrapfigure}

Since \textit{E.coli} cell membrane is negatively charged, adsorption of cells to walls might occur. We coated the channel walls with $0.1\%$ BSA solution dissolved in MMA to prevent it. BSA also has negative charges so that it will act as a blocking protein. Then, we add bacteria and seal the access holes with a glass coverslip preventing external flows in the experiment as shown in figure \ref{channel_diagram} c). We used an inverted microscope (Nikon TS100F) with a 40x/0.6 NA Plan Fluor objective to measure bacteria fluorescence and recorded it with a camera (Andor Zyla 2048 × 2048 \SI{}{\square\pixels}) at 10 fps, gain $4$, and 2x2 binning giving a resolution of \SI[per-mode = symbol]{0.32}{\micro\meter\per\pixels}. Since the PDMS thickness may vary, adjusting the correction ring of the objective to the appropriate dimensions is necessary. The focal plane will be on the edge of the channel, whose depth is approximately \SI{2}{\micro\meter}. Therefore we assume that the depth of the channel will not affect the observed dynamics.

\section{Image analysis}


This section describes how we analyzed the videos obtained from the experiments. Videos can be seen as just a 3-axis matrix with values $v_{ij}^t$ where $i,j$ are the pixel position indices and $t$ represents the frame number. The magnitude of $v_{ij}^t$ is the intensity of that pixel. The measured intensities consist of two sources, camera noise $n_{ij}^t$ and bacteria fluorescence $b_{ij}^t$. Our goal is to use $b_{ij}^t$ to measure different properties of the system. In figure \ref{video_histogram} it is possible to see a typical frame and histogram for the intensities $v_{ij}^t$.

\begin{figure}
	\centering
	\includesvg[scale=1]{imagenes/typical_video_frame}
	\caption[Typical video frame]{a) Example of one frame in a video and b) intensity histogram of the entire video. The video is saved in a 12-bit format so $v$ can take values from $0$ to $4096$ }
	\label{video_histogram}
\end{figure}


\subsection{Mask creation}

Masks are binarized images that determine the region of interest (ROI) for the experiment. In this case,  we look for the region where bacteria swim, bounded by the flat and sinusoidal walls. The procedure starts by creating an image $W$ from the original video that contains the max value reached for each pixel in the video $W_{ij}={\rm max}_{t\in[1,T_{max}]} v_{ij}^t$. Pixels that only represent noise will have maximum values $W_{ij}$ around the noise distribution, but the presence of a bacterium in a pixel at a given time will considerably increase the obtained maximum value at that pixel. Since bacteria mostly swim near walls, we can measure the contour of the walls as we see in figure \ref{mask} a). In order to binarize the image, we use the OpenCV python package whose thresholding function includes Otsu's binarization algorithm \cite{Grdiet2013BinarizationABSTARCT}. This method will automatically calculate a threshold for the image as a point between two intensity peaks. As we see in figure \ref{mask} b), the histogram has optimal conditions for this method. By doing this, we can detect the wall boundaries, and by filling all the pixels in between with ones, we can create a binary image with the region of interest. Finally, the tilting angle of the experiment can be measured as the angle between the x-axis and the flat wall. Thus the binarized image and the whole video can be rotated to produce a horizontal wall. The final result is displayed in \ref{mask} d).

\afterpage{%
\begin{figure}
	\centering
	\includesvg[scale=1]{imagenes/mask_example}
	\caption[Mask example]{a) Max pixel intensity image, created directly from the video. b) histogram of the max pixel intensity image $W$. The first peak of the histogram is associated with the noise distribution $n_{ij}^t$ and the second with $b_{ij}^t$. c) Binary mask derived from the max intensity image and d) the rotated, so the flat wall is horizontal. }
	\label{mask}
\end{figure}
}

The mask serves two primary purposes. The first one is to determine the actual dimensions of the channel. We manually enter the position of the peaks and valleys to the program. The amplitude $A$ is half of the mean vertical distance between a peak and a valley from the input locations, while the wavelength $\lambda$ is the horizontal separation between two consecutive peaks or valleys. The input has subpixel precision, so errors are only associated with mouse movement. Errors are not systematic because they are added randomly, so the method gives reasonable measures.  An automatic version will introduce many problems, as the border of the mask is not smooth, making many criteria not robust. 

The second purpose is to define the boundary of the experiment. We define the surfaces $B_{c},B_{f}$ to be \SI{4}{\micro\meter} thick from each wall into the bulk system. We call these surfaces the bands around the curved and flat walls, respectively.  Bands are used to measure the density near walls and decide if a bacterium is in contact with the wall.  In figure \ref{bands} there is an example showing both bands. The height of the bands was chosen to ensure that particles swimming in contact with the wall are entirely inside the band region. In some cases, bacteria will be swimming barely not in contact with the wall but still inside the band region. These cases are considered marginal since most of them will reach the wall one or two frames after.

\begin{wrapfigure}{r}{0.5\linewidth}
\centering
\includegraphics[width=\linewidth,angle=0]{imagenes/bandas22.10.21_A=3_L=21.png}
\caption[Bands example]{Image of a band for the same mask shown in figure \ref{mask} c).}
\label{bands}
\end{wrapfigure}

\subsection{Noise treatment}

As we said, the intensity values are given by $v_{ij}^t  = n_{ij}^t +  b_{ij}^t$. We are only interested in $b_{ij}^t$, and therefore, we want to minimize the effects of the noise. One important thing to consider is that the mean noise intensity is not the same for every pixel. In other words, the probability density function of noise intensity depends on space. There are two main reasons for this inhomogeneity. First, there are different noise levels at the PDMS and the MMA because out of focus bacteria contribute to the noise, and we can see in figure \ref{noise_img} a) and b) that the noise is higher on the liquid. Second, there is inhomogeneous illumination because we close the diaphragm to decrease noise intensity, only allowing light in the ROI to enter the camera, but the effect is more intense on the edges of the image. This effect is often less important than the first.

To deal with the dependence on space of $n_{ij}^t$, we exploit many properties of the data. We first approximate the noise distribution by just calculating the mean intensity of  $\bar{v} = \langle v_{ij}^t \rangle_{ijt}$ where the average is over space and time, as the lower indices indicate. Since for most pixels, the bacteria intensity $b_{ij}^t=0$ as bacteria occupy a small fraction of the ROI, $\bar{v}$ is slightly greater than the mean of $n_{ij}^t$ for different pixels. Now, we assume that every pixel satisfying $v_{ij}^t  < \bar{v} + 3 \sigma$ where $\sigma$ is the standard deviation of $v_{ij}^t$ do not include bacteria fluorescence, i.e. $b_{ij}^t =0$. The $v_{ij}^t$ values that satisfy the previous inequality are renamed as $V_{ij}^t$. The histogram of $V_{ij}^t$ is shown in figure \ref{noise_img} c). Then, we are only interested in the region with MMA and bacteria, so we estimate the mean of the noise in that region:

\begin{equation} \label{eq:noise_aproximation}
	\langle n_{ij}^t \rangle_{ijt} \approx {\rm mean_{ROI}}(V_{ij}^t).
\end{equation}

\afterpage{%
\begin{figure}[H]
	\centering
	\includesvg[scale=1]{imagenes/noise_example}
	\caption[Mean noise example]{ a) Mean noise image obtained averaging over time $V_{ij}^t$. The experiment displayed is one where this effect is clearly seen and is different from previous figures b) Histogram of $V_{ij}^t$ over two $20 \times 500$ \SI{}{\square\pixels} windows, one in PDMS and the other on MMA. The noise distribution on MMA is shifted to the right compared with the one on PDMS. Also, the difference is greater toward higher values, probably due to the influence of bacteria out of the focal plane. c) Histogram of $V_{ij}^t$ only for pixels in the ROI. The ${\rm mean_{ROI}}$ function calculates the mean over this distribution. d) Histogram of $B_{ij}^t$, the video that is used for future measurements. The segmented black line is on $B=0$.  }
	\label{noise_img}
\end{figure}
}

Here the ${\rm mean_{ROI}}$ function represents the mean of the values of $V_{ij}^t$ for pixels in any frame but inside the ROI defined by the mask. The resulting $\langle n_{ij}^t \rangle_t$ is then substracted to the video so now the intensities on the video are given by $B_{ij}^t  = b_{ij}^t + N_{ij}^t$, where $N_{ij}^t  = n_{ij}^t - {\rm mean_{ROI}}(V_{ij}^t)$. We affirm that any average in time and space of $B_{ij}^t$ will be for all effects equal to one done over only $b_{ij}^t$ since the matrix $N_{ij}^t $ averages 0 in the ROI. 



\subsection{Intensity analysis}

The properties of the $B_{ij}^t$ matrix give the possibility to measure mean bacteria intensity over time and space identically as measured with $ b_{ij}^t$. We are interested in how bacteria behave in the bands of each wall and how both walls compare. To do so, we consider $M_{ij} = \langle B_{ij}^t \rangle_t $. The image $M_{ij}$ is called the mean image of the experiment and is an indicator of where bacteria swam through. An example of $M_{ij}$ is shown in figure \ref{mean_image_and_profile} a). Then we consider the intensity near a wall $i_i$ as:

\begin{equation}
	i_i^y = \sum_{j \in B_y} M_{ij},
\end{equation}

where $B_y$ is the band of a specific wall $y$, then the definition of $i_i^y$ is the vertical sum of $M_{ij}$ over the band $B_y$. We normalize the intensity profiles by the mean of the flat wall intensity profile $\bar{i}^f = \langle  i_i^f\rangle_i $. 

\begin{equation}  
	\tilde{i}_i^y = \frac{i_i^y}{\bar{i}^f}.
\end{equation}

This normalization allows the comparison between different experiments respect to the standard of the flat wall. It also removes the problem of different bacteria intensities $b_{ij}^t$ due to fluorescence decay. 

The curved wall has a sinusoidal form, so it is reasonable to average $\tilde{i}_i^y$ over every period. If the experimental data has $N$ periods for the experiments whit amplitude $A$ and wavelength $\lambda$ then the intensity profile $I^y(x)$ of the wall $y$ with that shape is:

\begin{equation} \label{eq:Intensity profile}
	I^y(x) = \frac{1}{N} \sum_{n=1}^N \tilde{i}_{x+ n\lambda}^y.
\end{equation}

It is essential to clarify that the coordinate $x$ only takes values among one wavelength as the average is on every period and $x=0$ corresponds to the position of the valley. One set of profiles is displayed in figure \ref{mean_image_and_profile} b) as an example. The profile of the flat wall is obviously flat and equal to $1$ for all experiments, so it will never be plotted again. For simplicity, from now on we define $I(x)$ as the intensity profile of the curved wall, without the upper index $c$. Chapter 4 will profoundly discuss all the information and physics related to intensity profiles.


\begin{figure}
	\centering
	\includesvg[scale=1]{imagenes/mean_image_and_profile}
	\caption[Mean image and profile of an experiment]{a) Mean image $M_{ij}$ of a experiment. The accumulation on the walls can be seen. b) Normalized intensity $I(x)$ for both walls. The errorbars are the confidence interval of $95\%$ for the estimation of the mean $I(x)$ over the $N$ periods.}
	\label{mean_image_and_profile}
\end{figure}

\subsection{Bacteria tracking}

Tracking, in general, refers to determining the trajectories of objects in an image sequence. There are two main steps when doing the tracking: object detection and track creation. In the subsection, we will describe two different methods for doing detection and the tracking method based on linear assignment problems (LAP tracker). These methods are the ones that gave the most success during this MSc thesis, but other options were studied. All of the methods that will be discussed are already implemented on Trackmate, an open-source plugin for Fiji \cite{Tinevez2017TrackMate:Tracking}. 


We start by describing object detection. The specific objective is to determine bacteria position via the intensity $B_{ij}^t$. Then, a detection algorithm should be able to measure intensity variations indicative of the presence of bacteria. One possibility is to use the Laplacian of Gaussian detector (LoG) \cite{Kong2013AApplications,Sage2005AutomaticDynamics}. This method is easier to understand if the matrix $B_{ij}^t$ is thought of as a scalar field $B(x,y,t)$. We only know some field values, but we can calculate integrals and derivatives using standard numerical methods. As the detector's name indicates, we first convolve with a Gaussian and then calculate the Laplacian of the result. The equations for the method are:  

\begin{align}
	G(x,y;\sigma) &= \frac{1}{\sqrt{2\pi \sigma^2}} \exp\left(  -\frac{x^2+y^2}{2\sigma^2} \right), \\
	B^{gb}(x,y,t;\sigma) &= B(x,y,t) * G(x,y;\sigma), \\
	R(x,y,t;\sigma)  &= \nabla^2 B^{gb}(x,y,t;\sigma). \label{LoG:result}
\end{align}

Here $G(x,y;\sigma)$ is a Gaussian kernel that, when convolved ($*$ operation) with the image, produces a polished version $B^{gb}(x,y,t;\sigma)$ where $gb$ stands for gaussian-blurred. The reason for this smoothing effect of this convolution is that it acts as a low-pass filter for the field \cite{WaltzaAnMachines}, removing highly space-dependent noise contributions. This smoothing is useful because, in tracking, we do not average B spatially or temporally. The value of $\sigma$ is $r/\sqrt{2}$ where $r$ is the estimate bacteria radius. More importantly, in equation \ref{LoG:result} the Laplacian operator $\nabla^2$ is applied to obtain $R(x,y,t;\sigma)$. The result is that bacteria with maximum intensity in the center will also have a minimum negative laplacian there, so local minima in $R(x,y,t;\sigma)$ are bacteria centers. Locality, in this case, refers to a circle of the estimated radius $r$, so the method detects bacteria as blobs of that size. There are more generalized versions of LoG that allow the consideration of non-circular particles \cite{Kong2013AApplications}. This method is ideal for images with noise and particles with a maximum intensity at their center and decaying at a radius $r$, but it only detects circumferences, so it should not be used if it is necessary to know the exact shape of the particles.

\begin{wrapfigure}{l}{0.5\linewidth}
\centering
\includegraphics[width=\linewidth,angle=0]{imagenes/track_video_frame.PNG}
\caption[Tracking video frame]{Example of a typical frame for the binarized matrix $T_{ij}^t$.}
\label{tracking_video_frame}
\end{wrapfigure}


Another possibility to consider is binarizing the videos. A binarized image will have sharp variations of intensity, and if done correctly, will not lose bacteria in the process. In our case, we again use Otsu's binarization method into the matrix $B_{ij}^t$. The result is then used for the tracking, so we call it $T_{ij}^t$. An example of a typical frame of this tracking video is shown in figure \ref{tracking_video_frame}. Bacteria detection in  $T_{ij}^t$ is straightforward because considering all connected regions as a particle is enough. This detection method is known as the thresholding detector. One possible issue is that there is only one detection when multiple cells collide. In figure \ref{detection_method_comparison} there is a comparison with results for the two methods. The LoG detector works well with noise, which does not mean that it will fail with a binarized image. Proper binarization often helps with any detection method. Conversely, the thresholding detector adapts to bacteria form and does not make fake detections. This last point convinced us to use a thresholding detector for these experiments.

\begin{figure}
	\centering
	\includegraphics[width=\linewidth]{imagenes/detection_method_comparison.png}
	\caption[Detection method comparison]{ Zoom image of a video $T_{ij}^t$, and bacteria detection results displayed in green for a) LoG detector and b) Thresholding detector. LoG detector adds virtual detections in large bacteria, but the thresholding method will account for multiple bacteria collisions as only one particle. Methods give comparable results, but the best choice will depend on experiment properties. In our case, Otsu's binarization method has the conditions to perform the algorithm successfully, so the overall thresholding detector has better results than the LoG detector. }
	\label{detection_method_comparison}
\end{figure}

Now we deepen into the automatic tracking method described in Trackmate's user manual. There are many options for automatic tracking that use simple criteria such as the nearest neighbor assignment or overlapping criteria. It is not meriting to explain these methods as they only work under restricted conditions. Instead, the idea behind LAP trackers gives a general tool for tracking without overcomplications \cite{Jaqaman2008RobustSequences}. A linear assignment problem refers to determining the assignment matrix A that satisfies:

\begin{equation}
  A = {\rm argmin} \left(  \sum_{k,l} A_{kl}C_{kl} \right),
\end{equation}

where $C_{kl}$ is a cost matrix and $A_{kl}$ is a boolean matrix of 1 (link) and 0 (no-link) with the restriction that there is only one link for each row or column. To understand how this problem is used for tracking is only needed to inspect the cost matrix $C$. The simpler form of $C$ is to consider connections between detections of two frames $t$ and $t+1$. These frames will have $n$ and $m$ detections, respectively. Then $C$ is an $(n+m) \times (n+m)$ with four quadrants. 

\begin{itemize}
	\item The top left quadrant of size $n \times m$ has the cost of linking a detection on frame $t$ to one in $t+1$. The cost is $C_{kl} = (D_{kl}P_{kl})^2$ where $D_{kl}$ is the distance between the detections and $P_{kl} = 1 + \sum_f p_{kl}^f$ where $p_{kl}^f$ is a penalization for differences on the particle feature $f$ given by $p_{kl}^f = W_f \frac{|f_k - f_l|}{f_k + f_l}$, where $f_k$ is the value of the feature $f$ for particle $k$. Features that are useful with the thresholding detector are bacteria perimeter and area. The usage of penalization by features requires these characteristics to be held constant for there to be a connection. The magnitude of the feature penalization $W_f$ regulates the importance of this conservation. If $D_{kl}$ exceeds the double of the mean distance traveled in each frame, $C_{kl}$ is set to infinity
	\item Top right and bottom left quadrants are cost for not linking particles for frame $t$ and $t+1$ respectively. The cost values are set to $c = 1.05 \times {\rm max}(C_{kl})$ where the maximum is only taken on the top left quadrant and does not consider the infinite values.
	\item The bottom right quadrant is an auxiliary matrix in the Munkres and Khum algorithm used to solve the LAP problem \cite{Munkres1957AlgorithmsProblems}.
\end{itemize}

Considering the form of the cost matrix $C_{kl}$, depending on the quadrant of the connections, the assignment matrix $A_{kl}$ will connect detections from one frame to another or start/end tracks. Applying this to all frames will create the trajectories of the particles. A similar LAP can be considered for all the resulting tracks. In this case, the goal is to merge tracks, so the cost matrix includes distances between ends and starts of tracks. The cost is infinite if the end occurs at a time $T$ before the start. This process is called gap closing, as it allows to solve gaps in the tracks caused primarily due to detection failures involved in collisions. If particles separate from each other soon after the collision, this process will fix the errors. If they do not, the involved particles could be interchanged, causing wrong links, so the time $T$ should be kept as low as possible.  The LAP tracker as both frame-to-frame linking and gap closing works for general-purpose tracking, but Brownian motion is the best performing case for this method.

Nevertheless, we used a more specific version of the LAP tracker that considers the trajectories' properties in our experiments. The modification is simple but has profound effects. Based on particle trajectory, a prediction of where spots will be in the next frame is made. Instead of linking detections to each other, detections are linked with predictions based on the previous frame. The Kalman filter, also known as the linear quadratic estimation algorithm, is used to make predictions. Kalman filtering is a world in itself and has applications in robotics, navigation of vehicles,  geophysics, among others \cite{Auger2013IndustrialReview,Aanonsen2009TheReview}. Here we need to know that the Kalman filter considers bacteria's previous velocities to predict the future positions. Predictions allow gap closing differently. If two particles collide, there will only be one detection for two tracks. Then, one prediction will not have a link, but another prediction can be made for the next frame based on the previous. Predictions can fail to link up to $N_f=4$ times before the track is ended. If a track successfully encounters a detection before the $N_f$ failed attempts, the gap is closed. However, the predictions are not registered as intermediate positions, only truly measured detections. This method has two more parameters: the initial search radius $r_i=\SI{15}{\pixels}=\SI{4.8}{\micro\meter}$ and the search radius $r_s=\SI{10}{\pixels}=\SI{3.2}{\micro\meter}$. Both are the maximum allowed distances for frame-to-frame linking, with the difference that $r_i$ is only for new track initiation while $r_s$ is for linking considering predictions. The algorithm is briefly summarized as:

\begin{itemize}
	\item In the first frame, particles are linked to the second frame using standard particle to particle connections. The result is many starting tracks whose initial velocity can be measured, so predictions via Kalman filtering are now possible.
	\item For subsequent frames, the LAP is changed to prediction to particle linking, but the cost matrix structure does not change. New velocity measures will be considered in the Kalman filtering predictions.
	\item If a track fails to connect its prediction to a detection, new predictions will be made for future frames based on the current prediction. If this fails up to $N_f$ times, the track is terminated.
	\item Also, new particles could appear, so if a detection fails to find a track, the next frame, that detection will be considered in a separate LAP for tracking initiation. This LAP will be solved only with particles not in a track and after solving the LAP with predictions. This assigning order means that a track creation can not cause other tracks to end.
\end{itemize}

The algorithm has success in gap closing and frame to frame linking, bypassing problems caused by collisions. This affirmation only applies if particles have a roughly constant velocity and the search radius $r_s$ is more than the displacement between successive frames. The second point is important because when bacteria collide with a wall, they may completely stop, so a short $r_s$ may exclude these cases ending the track. All the results of tracking are left for chapter 4, meanwhile in figure \ref{tracking_examples} examples of trajectories are shown. In table \ref{table:image analysis parameters} all the relevant quantities for the methods described in this chapter are summarized.

\begin{figure}
	\centering
	\includesvg[scale=1]{imagenes/tracking_trajectories}
	\caption[Example of trajectories]{Example of trajectories obtained with the tracking method. The star represents the start of a track and the triangles the end of them. The mask is used as background to show the walls. Dimensions of the curved wall are shown at the bottom.}
	\label{tracking_examples}
\end{figure}


\begin{table}[!h]
   \centering
    \small
    \caption[Summary of all quantities used in the image analysis]{All of the quantities used for the image analysis, with their description and values. Parameters that are discussed but not include in the methods are not included in thise table. }
    \begin{tabularx}{\textwidth}{lXl}
    \hline\noalign{\smallskip}
         Quantity  & Description & Values   \\
    \noalign{\smallskip}\hline\noalign{\smallskip}
         $A$ & Amplitude of the sinusoidal curved wall. & \sim \SIlist[list-units=single, list-final-separator = {, }]{3;6;9;12}{\micro\meter} \\ 
         $\lambda$ & Wavelength of the sinusoidal curved wall & \sim \SIlist[list-units=single, list-final-separator = {, }]{21;24;27;30}{\micro\meter} \\
         $v_{ij}^t$ & Intensity of the raw data in the video. & \quad \\
         $\bar{v}$ & Mean of the matrix $v_{ij}^t$. & \quad \\
         $\sigma$ & Standard deviation of the matrix $v_{ij}^t$. & \quad \\
         $b_{ij}^t$ & Part of the signal $v_{ij}^t$ associated to bacteria fluorescence. & \quad \\
         $n_{ij}^t$ & Part of the signal $v_{ij}^t$ associated to the background noise. & \quad \\
         $W_{ij}$ & Image with the max value of $v_{ij}^t$ for all frames. After binarization is the mask that defines the ROI of the experiment. & \quad \\
         $B_c$ & Band of the curved wall. Is the surface \SI{4}{\micro\meter} thick starting from the wall boundary into the ROI.  & \quad \\
         $B_f$ & Band of the flat wall. Is the surface \SI{4}{\micro\meter} thick starting from the wall boundary into the ROI.  & \quad \\
         $V_{ij}^t$ & Video data considering only values that satisfy $v_{ij}^t  < \bar{v} + 3 \sigma$. The mean of this matrix over the ROI is used to eliminate noise contributions.  & \quad \\
         $B_{ij}^t$ & The raw data $v_{ij}^t$ minus ${\rm mean_{ROI}}(V_{ij}^t)$. Averages of $B_{ij}^t$ are equal to one done exclusively with $b_{ij}^t$.  & \quad \\
         $M_{ij}$ & Mean image obtained as $\langle B_{ij}^t \rangle_t$  & \quad \\
         $i_{i}$ & Intensity near a specific wall, averaged over the vertical direction in the respective wall.  & \quad \\
         $I_{i}$ & Characteristic intensity profile in one period corresponding to the periodical average of $i_i$ (see equation \ref{eq:Intensity profile}).   & \quad \\
         $r_i$ & Initial search radius for the LAP tracker. Radius $r_i$ is used for particle-particle linking at the beggining of a track.   & \SI{4.8}{\micro \meter} \\
         $r_s$ & Search radius for a current tracking the LAP tracker. Radius $r_s$ is used for particle-prediction linking.    & \SI{3.2}{\micro \meter} \\
         $N_f$ & Number of allowed failures for particle-prediction linking before the track is ended. & 4 \\
    \hline\noalign{\smallskip}
    \end{tabularx}
    \label{table:image analysis parameters}
\end{table}

\chapter{Numerical models and simulations}

\section{Theoretical framework}

This section explains details of the numerical model considered and the criteria used for its election. We consider an agent-based model of spherical active particles with overdamped dynamics in a 2-dimensional representation. 

\subsection{Agent-based models} 

In biophysics, models can be sorted into two classes, agent-based, and continuum models. Agent-based models simulate the dynamics of cells individually, offering accurate system descriptions. Depending on the complexity of the model, they can be costly in a computational sense. On the other hand, continuum models consider a coarse-grained description with density and velocity fields. We work in a low-density regime, so a continuum description is meaningless. Therefore, we will use agent-based models. For simplicity, we consider cells with spherical shapes. In reality, the cell shape depends on the growth phase of the bacteria culture, but \textit{E.coli}'s body is a spherocylinder with typical dimensions of \SI{0.5}{\micro\meter} in radius and \SI{2}{\micro\meter} in length.

\subsection{Overdamped dynamics}

To describe the dynamics of single cells, we invoke Newton's second law: 

\begin{equation}
	m\ddot{\textbf{r}} = \sum \textbf{F} = \textbf{F}_{\text{hydro}} + \textbf{F}_{\text{flag}} + \textbf{F}_{\text{brown}} + \textbf{F}_{\text{coll}} ,
\end{equation}

where we define four different force sources, $\textbf{F}_{\text{hydro}}$ the hydrodynamic friction produced by the fluid, $\textbf{F}_{\text{flag}}$ the force produced by the flagella, $\textbf{F}_{\text{brown}}$ the Brownian force due to collisions with the liquid molecules, and $\textbf{F}_{\text{coll}}$ produced due to cell-cell and cell-surface collisions. We can deduce an important fact widely accepted in microscopic systems with the following analysis. To analyze the role of inertia, let's imagine we have isolated \textit{E.coli} bacterium that suddenly stops swimming, so $\textbf{F}_{\text{flag}} = \textbf{F}_{\text{coll}} =0$. Although the influence of $\textbf{F}_{\text{brown}}$ is essential, its effect is stochastic with zero mean. If we repeat the situation many times and average the ensemble, its contribution is expected to be zero. Therefore, for the moment, we will neglect it. Then we are left with the equation:

\begin{equation}
	m\ddot{\textbf{r}} = \textbf{F}_{\text{hydro}} = -\gamma \dot{\textbf{r}} = - 6 \pi R \eta \dot{\textbf{r}},
\end{equation}

where we used Stokes' law to calculate the drag force coefficient $\gamma$ with $R=$ \SI{0.5}{\micro\meter} being the particle radius and $\eta=$ \SI{d-3}{\pascal\cdot\second} the viscosity. Since the mass of an \textit{E.coli} is $m=$ \SI{d-12}{\gram}, the characteristic time of slowing down is given by $ m/(6 \pi R \eta) \approx $ \SI{d-7}{\second}. This means that in the timescale of \SI{1}{\micro\second} the bacteria should have stopped. We record at 10 fps in our experiments, so the detention is immediate for our time resolution. This means that we are working in the overdamped limit, where inertia is negligible, and so it is correct to simplify the dynamics as:

\begin{equation} \label{eq:overdamped_model}
	\gamma \dot{\textbf{r}} = \textbf{F}_{\text{flag}} + \textbf{F}_{\text{brown}} + \textbf{F}_{\text{coll}}.
\end{equation}

\subsection{Rotational diffusion}

We considered two different effects of the fluid in the bacterial dynamics. First, $\textbf{F}_{\text{hydro}}$ is a phenomenological dynamical drag force that represents the average effect of the liquid on an object moving through the fluid, and second, $\textbf{F}_{\text{brown}}$ of stochastic kind accounting for the thermal fluctuations. The latter produces a mean square displacement of $\langle r^2 \rangle (t) = 4 D t $ where $D$ is the diffusion constant given by the fluctuation-dissipation theorem \cite{Soto2016KineticPhenomena}:

\begin{equation}
	D = \frac{k_bT}{\gamma} \approx \SI[per-mode = symbol]{1.5d-1}{\square\micro\meter \per \second},
\end{equation}
 
where $k_b$ is the Boltzmann constant, $T$ the temperature of the fluid. The same value for $\gamma$ was used. This diffusion constant is purely translational. Nevertheless, the collisions between cells and water molecules also exert torques on the bacteria, meaning that the orientation of swimming is subject to an equivalent stochastic process. This process has a rotational diffusion constant $D_r$ independent of $D$. For smooth-swimming \textit{E.coli} it has been measured at $D_r=$ \SI[per-mode = symbol]{0.057}{\square\radian \per \second} \cite{Drescher2011FluidScattering}. We can perceive the importance of rotational diffusion by neglecting collisions in equation \eqref{eq:overdamped_model} and calculating the mean square displacement as described in \cite{Lauga2020TheMotility}.

\begin{equation} \label{eq:flagellar force}
	 \dot{\textbf{r}} =\frac{\textbf{F}_{\text{flag}} + \textbf{F}_{\text{brown}} }{\gamma} = u \textbf{p} + \textbf{F}^*_{\text{brown}},
\end{equation}

where $u=$ \SI[per-mode = symbol]{20}{\micro\meter \per \second} is the mean speed of bacteria, $\textbf{p}$ the vector of orientation, and $\textbf{F}^*_{\text{brown}}$ is just the force with the coefficient $\gamma$ absorbed. Integrating equation \eqref{eq:flagellar force} in time we obtain:

\begin{equation} \label{eq:positon}
	\textbf{r}(t) - \cancelto{0}{\textbf{r}(0)}  = \int_0^t [u \textbf{p}(t^\prime) +  \textbf{F}^*_{\text{brown}}(t^\prime)  ]dt^\prime ,
\end{equation}

Taking the dot product of \eqref{eq:flagellar force} and \eqref{eq:positon} we obtain an equation for the time derivative of the square displacement.

\begin{equation} \label{eq:derivative of msd}
	\dot{\textbf{r}} \cdot \textbf{r} = \frac{1}{2}\frac{d}{dt} (r^2) = \int_0^t  [u^2 \textbf{p}(t^\prime) \cdot \textbf{p}(t) + \textbf{F}^*_{\text{brown}}(t) \cdot \textbf{F}^*_{\text{brown}}(t^\prime)]dt^\prime + \text{irrelevant terms}.
\end{equation}


All the terms in the right side of equation \eqref{eq:derivative of msd} are stochastic and vary from cell to cell. Nevertheless, their average for all particles has properties that allow progress in the calculation. For example, the terms cataloged as irrelevant consider dot products of two quantities completely uncorrelated, the director vector $\textbf{p}$ and the force $\textbf{F}^*_{\text{brown}}$. Therefore after averaging on the ensemble of cells, the average dot product is zero, and we are left with: 

\begin{equation} 
	\frac{d}{dt} \langle r^2 \rangle &= 2 \int_0^t  \langle u^2 \textbf{p}(t^\prime) \cdot \textbf{p}(t) + \textbf{F}^*_{\text{brown}}(t) \cdot \textbf{F}^*_{\text{brown}}(t^\prime) \rangle dt^\prime.
\end{equation}


The quantity $\langle\textbf{p}(t^\prime) \cdot \textbf{p}(t)\rangle$ is the mean time correlation of the vector director \textbf{p} and depends on the rotational diffusion coefficient as $e^{-2D_r|t-t^\prime|}$ \cite{Lauga2020TheMotility}. On the other side $\textbf{F}^*_{\text{brown}}$ is assumed to be an uncorrelated noise, that satisfies $\langle \textbf{F}^*_{\text{brown}}(t) \cdot \textbf{F}^*_{\text{brown}}(t^\prime) \rangle = 2D\delta(t^\prime -t)$. Therefore, integrating over both $t$ and $t^\prime$ we obtain:


\begin{equation} 
    \langle r^2 \rangle (t) = \frac{u^2}{D_r}\left( t + \frac{e^{-2D_rt}}{2D_r} - \frac{1}{2D_r} \right) + 4Dt
\end{equation}

At short times,  $t \ll D_r^{-1} =$ \SI{18}{\second} the exponential can be expanded in a Taylor series $e^{-2D_rt}\approx 1 - 2D_rt + 2(D_r t^2) + \mathcal{O}(t^3)$ giving $\langle r^2 \rangle (t)  \approx (ut)^2$, which means that at short times particles swim is straight lines. Bacteria swimming is more important than diffusion even at our lowest time scale of $t=$ \SI{0.1}{\second}, as in that case $4Dt$ is two order of magnitude lower than $(ut)^2$. In the other case $t \gg D_r^{-1}$ the mean square displacement is $(u^2/D_r+4D)t$. Adding the translational diffusion discussed previously, we deduce an effective diffusion constant for long times $D_{\text{eff}}$ given by:

\begin{equation}
    D_{\text{eff}} = D + \frac{u^2}{4D_r}
\end{equation}

For the typical values mentioned, $D_{\text{eff}}\approx$ \SI[per-mode = symbol]{2d3}{\square\micro\meter \per \second} which is four orders of magnitude larger than $D$. This means that bacteria swimming makes translational Brownian motion irrelevant compared to the effects of rotational diffusion. We conclude that $\textbf{F}_{\text{brown}}$ can be set to zero for simplicity without losing relevant dynamics. We are left with the equation:

\begin{equation} \label{eq:final_model}
    \dot{\textbf{r}} = u\textbf{p} + \frac{1}{\gamma}\textbf{F}_{\text{coll}} = u\textbf{p} + \textbf{F}^*_{\text{coll}} ,
\end{equation}

where $\textbf{F}^*_{\text{coll}}$ absorbs the drag coefficient $\gamma$ and hence has units of speed.

\subsection{Alignment with the wall}
\label{section:steric alignment}

We observe that bacteria interacting with the curved and flat walls, suffer a torque that aligns them with the wall \cite{Bianchi2017HolographicBacteria}. This alignment has its origin on steric forces. If we consider equation \eqref{eq:final_model} and take the dot product with the vector perpendicular to the surface  $\hat{\textbf{n}}_w$ at the point of contact, we obtain:

\begin{equation} \label{eq:steric_force}
   \cancelto{0}{\dot{\textbf{r}} \cdot \hat{\textbf{n}}_w}  = u\textbf{p} \cdot \hat{\textbf{n}}_w + \textbf{F}^*_{\text{coll}} \cdot \hat{\textbf{n}}_w ,
\end{equation}

where $\dot{\textbf{r}} \cdot \hat{\textbf{n}}_w = 0$ is imposed as bacteria do not cross the surface. Since the collision force with the wall $\textbf{F}^*_{\text{coll}}$ is exclusively perpendicular to the wall, equation \eqref{eq:steric_force} gives the magnitude of the collision force as $\gamma u \textbf{p} \cdot \hat{\textbf{n}}_w$, where $\gamma$ reappeared because we are calculating the force. Then we can write the equation for the angle of swimming $\theta$ considering that the inertia of the cell is negligible and so that the sum of torques must be zero. Simplifying the system, we consider the torques respect to the center of mass as the rotational drag and the torque exerted by the wall. Other torques can be considered for more complete models to explain observations such as circular trajectories and longer residence times in the wall \cite{Lauga2006SwimmingBoundaries, Sipos2015HydrodynamicWalls}. For this thesis we decided to work with a simple model because these effects are not required to explain the abandonment of the curved wall by bacteria. Using equation \eqref{eq:steric_force}, 

\begin{equation} \label{eq:deduction_of_K}
    0 = - \gamma_r \dot{\theta} \hat{\textbf{z}} - u \gamma (\textbf{p} \cdot \hat{\textbf{n}}_w) (\textbf{r}_{cm} \times \hat{\textbf{n}}_w),
\end{equation}

where $\gamma_r$ is the rotational drag coefficient and $\textbf{r}_{cm}$ the vector from the center of mass to the point of contact, as we are calculating the torque with respect to the center of mass. Therefore, $\textbf{r}_{cm} \times \hat{\textbf{n}}_w = L_{cm} (\textbf{p} \cdot \hat{\textbf{t}}_w) \hat{\textbf{z}}$ where $\hat{\textbf{t}}_w$ is the tangential vector to the wall in the point of contact that satisfies $\hat{\textbf{z}} = \hat{\textbf{t}}_w \times \hat{\textbf{n}}_w$  and we used $\textbf{r}_{cm} = L_{cm} \textbf{p} $ with $L_{cm}$ is the distance between the center of mass and the point of contact taking into consideration the contribution of the flagella to the center of mass. The elements involve in these equations are shown in figure \ref{cp_wall_diagram}. We are left with the equations:

\begin{align}
    \textbf{p} &= \cos{\theta}\hat{\textbf{x}}+\sin{\theta}\hat{\textbf{y}}, \\
    \dot{\theta} &= -K (\textbf{p} \cdot \hat{\textbf{t}}_w)  (\textbf{p} \cdot \hat{\textbf{n}}_w) \Gamma(\textbf{r}, \textbf{p}),
    \label{eq:wall alignment}
\end{align}

where $K\equiv u L_{cm}\gamma / \gamma_r$. The scalar $K$ controls the intensity of the alignment and in ref. \cite{Bianchi20193DInterface} a fit of experimental data yields $K=$ \SI[per-mode = symbol]{4.9}{\radian \per \second}. In our study we consider $K$ as a free parameter with values between \SIlist[per-mode = symbol, list-units=single]{0;7}{\radian\per\second}. This is because we are considering the more complicated case of a curved wall. Also, $\Gamma(\textbf{r}, \textbf{p})$ is a step function that indicates if the bacteria is in contact with the wall or not. It is equal to $1$ when the distance from the center of the particle $\textbf{r}$ to the closest point of the wall is smaller than one radius, and its swimming direction points towards the wall i.e. $\textbf{p} \cdot \hat{\textbf{n}}_w < 0$ indicating the cell is going into the wall and not away from it. Otherwise $\Gamma$ equals zero.  Equation \eqref{eq:wall alignment} will align the vector $\textbf{p}$ with the $\pm\hat{\textbf{t}}_w$ depending on the sign of $\textbf{p} \cdot \hat{\textbf{t}}_w$. This effect is only considered for the flat and curved wall as the frontal wall is already taken into account because simulations are two-dimensional.

\begin{wrapfigure}{r}{0.5\linewidth}
\centering
\includesvg[width=\linewidth,height=4cm]{imagenes/cp_wall_diagram}
\caption[Diagram of the components involve in the wall allignment]{Diagram of the components involve in the wall alignment. $O$ represents the origin of coordinates.}
\label{cp_wall_diagram}
\end{wrapfigure}

We calculate the point of contact as the closest point in the walls to the cell by dividing the interval $[x-R, x+R]$ in 300 points where $x = \textbf{r} \cdot \hat{x}$. Then, we calculate the distance between the cell and the position obtained with the parametric definition of the wall for each point. The point $\textbf{r}_w$ with the lowest distance is the closest with a precision of \SI{3d-3}{\micro\meter}. This ``brute force" algorithm works because a point outside of the interval $[x-R, x+R]$ is not in contact with the cell, and it is convenient because the distance to the wall as a function of the $x$-axis has many local minima.

% \vspace{1cm}

\section{Simulations}

Equations \eqref{eq:final_model}--\eqref{eq:wall alignment} summarize the physics involved in the description of the system. We are only missing two aspects; rotational diffusion and the details of $\textbf{F}_{\text{coll}}^*$. This section first explains how we implement these phenomena, summarize the model, and finally explain how the simulations are performed.
 
\subsection{The collision force}

$\textbf{F}^*_{\text{coll}}$ is the force produced by the collision of cells with other cells or the wall, but divided by the drag coefficient $\gamma$. We considered it as an elastic force for both cell-cell and cell-wall collisions. We parameterize the force $\textbf{F}^{*,i}_{\text{coll}}$ that the $i$-th particle experiments through the following: 

\begin{equation}
   \textbf{F}^{*,i}_{\text{coll}} &= \sum_{w} k_{\text{wall}} (R-d_{iw}) \hat{\textbf{d}}_{iw}\Theta(R-d_{iw}) + \sum_{j \in \mathbb{NN}_i} k_{\text{cell}} (2R-d_{ij}) \hat{\textbf{d}}_{ij}\Theta(2R-d_{ij}),
\end{equation}
 
where $d_{iw}$, $\hat{\textbf{d}}_{iw}$ represent the distance and the unit vector between the particle $i$ and the closest point of the wall $w$, and $d_{ij}$, $\hat{\textbf{d}}_{ij}$ are the same but between the particles $i,j$ where $j$ is part of $\mathbb{NN}_i$ which is the set of nearest neighbors of the particle $i$. $\Theta$ is the usual Heaviside function. Finally the parameters $k_{\text{wall}}$, $k_{\text{cell}}$ are the intensities of these forces, and have units of \SI[]{}{\per\second}. We consider $k_{\text{wall}}=$ \SI[]{1d3}{\per\second}, which means that for $R-d_{iw}=$ \SI{2d-2}{\micro\meter} the magnitude of the interaction with the wall is equal to $u$, so it is impossible for bacteria to go through the surface. The parameter $k_{\text{wall}}$ is not related to $K$ it only defines the distance at which the elastic force is enough to repel bacteria. Meanwhile $k_{\text{cell}}$ will be considered as zero, meaning there are no cell-cell interactions. The explanation for such consideration is on section \ref{section: clustering}.

\subsection{Rotational diffusion}

 Rotational diffusion is naturally added to \eqref{eq:wall alignment} as a Gaussian white noise $\eta_i(t)$ \cite{Digregorio2018FullSeparation,Caporusso2020Motility-InducedSystem} meaning it has zero mean and its completly uncorrelated in time and between particles, $\langle \eta_i(t)\eta_j(t^\prime)  \rangle \propto \delta_{ij}\delta (t-t^\prime)$. This white noise will change randomly the direction of swimming $\textbf{p}$ of the cell. We will go into more detail in the following sections.
 
\subsection{Final model}

The final set of equations that are used in the model are:

\begin{align}
    \label{eq:dynamics of position}
    \dot{\textbf{r}}_i &= u\textbf{p}_i + \textbf{F}^{*,i}_{\text{coll}}, \\
    \textbf{p}_i &= \cos{\theta_i}\hat{\textbf{x}}+\sin{\theta_i}\hat{\textbf{y}}, \\
    \label{eq:dynamics of angle}
    \dot{\theta}_i &= -K (\textbf{p}_i \cdot \hat{\textbf{t}}_w)  (\textbf{p}_i \cdot \hat{\textbf{n}}_w) \Gamma(\textbf{r}_i, \textbf{p}_i) + \eta_i(t), \\
    \label{eq:elastic force}
    \textbf{F}^*_{\text{coll}} &=   k_{\text{wall}} (R-d_{iw}) \hat{\textbf{d}}_{iw}\Theta(R-d_{iw}) + \sum_{j \in \mathbb{NN}} k_{\text{cell}} (2R-d_{ij}) \hat{\textbf{d}}_{ij}\Theta(2R-d_{ij}).
\end{align}

All of the quantities used in these equations and on the model are described in table \ref{table:model parameters}. These equations are written for the dynamics of a circular particle $i$, involving other entities such as the wall and the nearest neighbors. Equations \eqref{eq:dynamics of position}--\eqref{eq:elastic force} belong to the class of Langevin equations due to their stochastic nature.

% Tabla del modelo
\begin{table}[!h]
   \centering
    \small
    \caption[Summary of the quantities used in the simulations]{Quantities used for the model, with their description and values if adequate. For parameters whose value changes, that column will have a $-$ symbol and in chapter 4, results will have that value specified. }
    \begin{tabularx}{\textwidth}{lXl}
    \hline\noalign{\smallskip}
         Variable  & Description & \quad   \\
    \noalign{\smallskip}\hline\noalign{\smallskip}
         \textbf{r} & Position of the particle $i$. & \quad \\ 
         \textbf{p} & Direction of swimming of the particle $i$, defined by the angle $\theta_i$. & \quad \\
         $\eta$ & White noise associated with rotational diffusion. & \quad \\
         $\hat{\textbf{t}}_w$ & Unitary vector tangent to the wall in the closest point to the cell $i$. & \quad \\
         $\hat{\textbf{n}}_{w}$ & Unitary vector normal to the wall in the closest point to the cell $i$. The normal vectors points away from the wall. & \quad \\
         $\hat{\textbf{d}}_{iw}$ & Unitary vector pointing from the closest point in the wall to the cell $i$. & \quad \\
         $d_{iw}$ & Distance between the cell $i$ and the closest point of the wall. & \quad \\
         $\mathbb{NN}$ & Set of nearest-neighbors of the particle $i$. & \quad \\
         $\hat{\textbf{d}}_{ij}$ & Unitary vector pointing from the neighbor $j$ to the cell $i$. & \quad \\
         $d_{ij}$ & Distance between the cell $i$ and the neighbor $j$. & \quad \\
    \hline\noalign{\smallskip}
        Parameter  & Description & Value   \\
    \noalign{\smallskip}\hline\noalign{\smallskip}
        $\rho$ & Density of cells. This density relates to the number of particles seen in the focal plane. & \SI[per-mode = symbol]{3d-3}{\cells \per \square\micro\meter} \\
        $u$ & Swimming speed. & \SI[per-mode = symbol]{20}{\micro\meter\per\second} \\
        $R$ & Radius of the cells. & \SI{0.5}{\micro\meter} \\
         $K$ & Magnitude of the aligment with the wall. & - \\ 
         $D_r$ & Rotational diffusion coefficient. & - \\ 
         $k_{\text{wall}}$ & Elastic constant for cell-wall collisions. & \SI{1d3}{\per\second}\\ 
         $k_{\text{cell}}$ & Elastic constant for cell-cell collisions. & 0 \\
         $\Delta t$ & Time step for the integration of the equations. & \SI{d-3}{\second} \\
         $\Delta t_r$ & Time step for the recording of data. & \SI{d-1}{\second} \\
         $T$ & Time duration of the simulations. & \SI{1200}{\second} \\
    \hline\noalign{\smallskip}
    \end{tabularx}
    \label{table:model parameters}
\end{table}

Quantities with the lower index $w$ depend on the wall. Simulations have a flat and a curved wall with amplitude $A$ and wavelength $\lambda$. The parametric definition of the walls positions are $y=y_f$ for the flat and $y=y_c+A\sin{\frac{2\pi x}{\lambda}}$ for the curved wall. The values of $y_f, y_c$ are so that the mean distance between walls is \SI{100}{\micro\meter}. Also, periodic boundary conditions will be applied. To avoid problems with discontinuities in the wall, the length of the channel is adjusted to have an integer number of wavelengths and be greater than \SI{300}{\micro\meter}. The total particle number is adjusted so that all simulations have the same particle density $\rho$. These values are similar to the experiments.

\newpage

\subsection{Numerical integration}

Typically, there are many options to solve a differential equation numerically. In this case, the equations are somewhat simple, so it is tempting to integrate \eqref{eq:dynamics of position} and \eqref{eq:dynamics of angle} with Euler's method. That integration method is a first-order method which means that during a step $\Delta t$ of integration, the integrands are taken as constants. Therefore we would obtain:

\begin{align}
    \textbf{r}_i(t+\Delta t) &=  \textbf{r}_i(t) + \Delta t[u \textbf{p}_i(t) + \textbf{F}^*_{\text{coll}}(t)], \\
    \label{eq:wrong integration}
    \theta_i(t+\Delta t) &=  \theta_i(t) + \Delta t[- K (\textbf{p}_i(t) \cdot \hat{\textbf{t}}_w)  (\textbf{p}_i(t) \cdot \hat{\textbf{n}}_w) \Gamma(\textbf{r}_i(t), \textbf{p}_i(t)) + \eta_i(t)].
\end{align}

However, there is a problem with the last term of equation \eqref{eq:wrong integration}, which relies on the timescale in which $\eta$ acts. Remember that $\eta$ represents the torque produced by collisions with the liquid molecules. These thermal fluctuations have a characteristic time between collisions of $\sim$\SI{1.9d-13}{\second} \cite{Soto2016KineticPhenomena}. That means we have two options, either we consider a minimal time step $\Delta t$ so the hypothesis that $\eta$ is constant in the interval is true, or we treat $\eta$ separately. The latter is the best option, as $\eta$'s timescale is much smaller than all the others present in the system. The correct equations are:

\begin{align}
    \textbf{r}_i(t+\Delta t) &=  \textbf{r}_i(t) + \Delta t[u \textbf{p}_i(t) + \textbf{F}^*_{\text{coll}}(t)], \\
    \theta_i(t+\Delta t) &=  \theta_i(t) - K \Delta t  (\textbf{p}_i(t) \cdot \hat{\textbf{t}}_w)  (\textbf{p}_i(t) \cdot \hat{\textbf{n}}_w) \Gamma(\textbf{r}_i(t), \textbf{p}_i(t)) \nonumber \\
     &\ \ \ + \int_t^{t+\Delta t}\eta_i(t^\prime)dt^\prime .
\end{align}


Then a new problem arises, how do we calculate the integral of $\eta$. We can find the answer on stochastic calculus, and here we describe one of the possible demonstrations. First, we treat $\eta$ as what it is, a discrete function representing all the collisions with liquid molecules.

\begin{equation}
    \int_t^{t+\Delta t}\eta(t^\prime)dt^\prime = \sum_{n=1}^N \phi\eta_n,
\end{equation}

where $\eta_n$ is proportional to the angle displacement produced in a certain collision $n$. The magnitude of these rotations is contained in $\phi$ so that we can treat them as standard normal Gaussians. As we stated, these collision displacements are also uncorrelated. Therefore, we can use that the distribution of the sum of two gaussian uncorrelated variables, $x\sim\mathcal{N}(\mu_x,\sigma_x^2)$ and $y\sim\mathcal{N}(\mu_y,\sigma_y^2)$, is given by $\mathcal{N}(\mu_x+\mu_y,\sigma_x^2+\sigma_y^2)$ and by induction obtain:

\begin{equation}
    \sum_{n=1}^N \eta_n \sim \mathcal{N}(0,N) = \sqrt{N}\mathcal{N}(0,1),
\end{equation}

were the last equality is in the sense of probability density distribution. Then, the total number of collisions $N$ is proportional to $\Delta t$, for instance $N=\alpha \Delta t$, where $\alpha$ is a rate of collisions. Then we can write:

\begin{align}
    \textbf{r}_i(t+\Delta t) &=  \textbf{r}_i(t) + \Delta t[u \textbf{p}_i(t) + \textbf{F}^*_{\text{coll}}(t)], \\
    \label{eq:correct integration}
    \theta_i(t+\Delta t) &=  \theta_i(t) - K\Delta t (\textbf{p}_i(t) \cdot \hat{\textbf{t}}_w)  (\textbf{p}_i(t) \cdot \hat{\textbf{n}}_w) \Gamma(\textbf{r}_i(t), \textbf{p}_i(t)) \nonumber \\
     &\ \ \ + \sqrt{\phi^2\alpha\Delta t} \eta(t).
\end{align}

Here $\phi$ is in \SI{}{\radian} and $\alpha$ in \SI{}{\per\second} so $\phi^2\alpha$ has units of \SI{}{\square\radian\per\second}, the same as the rotational diffusion coefficient. Therefore, we naturally obtained the diffusion coefficient, normally defined as $2D_r\equiv\phi^2 \alpha$. As already mentioned, for smooth swimmer \textit{E.coli} in liquids it has been measured that $D_r=$ \SI[per-mode = symbol]{0.057}{\square\radian \per \second}. Nevertheless we consider $D_r$ as a free parameter with values between $D_r=$ \SIrange[per-mode = symbol, range-units=single
]{0.001}{0.05}{\square\radian \per \second} as is expected to be lower in our case, due to the constraints of the walls. Therefore $K$ and $D_r$ are the free parameters of the model.

The result of equation \eqref{eq:correct integration} is really important for the consistency of the numerical integration, since the equation $\langle \eta(t)\eta(t^\prime)  \rangle &= \delta (t-t^\prime)$ implies that $\eta$ has units of \SI{}{\second^{-1/2}} so multiplying by $\Delta t$ instead of $\sqrt{\Delta t}$ is dimensionally wrong.


\subsection{Intensity profiles in simulations}


In experiments we construct intensity profiles using the intensity of bacteria fluorescence. In simulations there is no such thing, we can only measure bacteria positions $\textbf{r}_i$. Therefore, we have to measure the mean bacteria density. For each particle in the band of a wall $w$ we can assign a interval in the $x$-axis defined as $[x-\Delta x/2,x+\Delta x/2]$ where $x_i = \textbf{r}_i \cdot \hat{x} - x_{\text{valley}}$ is contained. Here $x_{\text{valley}}$ is the position of the nearest valley to the particle, meaning $x$ is on a interval of one wavelength $\lambda$ and $x=0$ is the position of the valley. This defines the mean bacteria density over a period $n^w(x)$ as the count of bacteria that were in contact with the wall $w$ in the interval defined by $x$. 

Obviously $n^w(x)$ is not the same as intensity profiles. To create a comparable quantity, we convolve $n^w(x)$ with a Gaussian function $G(x)=\exp(-x^2/(2R^2))$. We call the result the intensity near a wall $w$ for the simulation:

\begin{equation}
    i^w_{\text{sim}}(x) = n^w(x) * G(x).
\end{equation}

This treatment means we consider particles having a Gaussian intensity in space. This is not equal to the intensities measured in experiments because bactearia are not spherical. Nevertheless, treating intensity not spherically in simulations would lead to inconsistencies. The amplitude of the Gaussian does not matter as we will normalize by the mean accumulation in the flat wall. If the mean intensity on the flat surface is $\bar{i}^f_{\text{sim}} = \langle  i_{\text{sim}}^f(x)\rangle_x $. We define the normalized intensity profile in the curved wall as:

\begin{equation} \label{eq:Intensity profiles in simulations}
    I_{\text{sim}}(x) = \frac{i_{\text{sim}}^c(x) }{\bar{i}^f_{\text{sim}}}
\end{equation}

The normalized intensity profiles of simulations $I_{\text{sim}}(x) $ will be compared to the experimental result $I(x)$ in the chapter 4.

\newpage

\subsection{Algorithm}

The algorithm used in the simulations is described in the following steps.

\begin{itemize}
    \item[1.] Create a random initial condition for all the particles positions and swimming directions. If a particle is out of boundaries or in contact with a wall, its initial conditions are generated again. Then start iterating the time steps.
    \item[2.] Update the closest point of the wall for all particles close to the walls.
    \item[3.] Integrate the equation for $\theta_i$.
    \item[4.] Determine the $\mathbb{NN}$ set for all particles.
    \item[5.] Calculate the force $\textbf{F}^*_{\text{coll}}$.
    \item[6.] Integrate the equation for $\textbf{r}_i$.
    \item[7.] Every time interval $\Delta t_r$ record the relevant data. This avoids high time correlations between measurements.
    \item[8.] Stop when a time $T=$ \SI{1200}{\second} has passed.
\end{itemize}

The algorithm was implemented in a C++ program with object oriented programming. In figure \ref{sim trajectories} we display trajectories for the parameters $K=$ \SI[per-mode = symbol]{5}{\radian\per\second}, $D_r=$ \SI[per-mode = symbol]{0.05}{\radian\per\second} and $k_{\text{cell}}=0$.

\begin{figure}[H]
	\centering
	\includesvg[scale=1]{imagenes/sim_trajectories}
	\caption[Example of trajectories for the simulations]{Example of trajectories  obtained with the simulations for the parameters $K=$ \SI[per-mode = symbol]{5}{\radian\per\second}, $D_r=$ \SI[per-mode = symbol]{0.05}{\square\radian\per\second} and $k_{\text{cell}}=0$. The star represents the start of each track and the triangles the end of them. The values of $A$ and $\lambda$ are included on the top of the plot. Trajectories last for \SI{10}{\second}.}
	\label{sim trajectories}
\end{figure}

\chapter{Results}

This chapter is dedicated to the results obtained both in experiments and with the model. We start analyzing intensity as an indirect measurement of mean density accumulation, proving there is a transition on the accumulation near walls. Results of tracking explain further this observations.  

\section{Intensity profiles}

\afterpage{%
\begin{figure}[H]
	\centering
	\includesvg[width=\linewidth]{imagenes/experimental_profiles}
	\caption[Experimental intensity profiles]{Experimental intensity profiles for a) $\lambda= $\SI{30}{\micro\meter} and b) $\lambda=$ \SI{24}{\micro\meter} with three different amplitudes. The colors were chosen so that curves associated with similar values of $A$ share color. The amplitudes that are shown allow to see the transition. Errorbars are the confidence interval of $95\%$ for the estimation of the mean $I(x)$. All other experimental intensity profiles are in appendix B. }
	\label{experimental profiles}
\end{figure}
}

We consider the normalized mean intensity profiles $I(x)$ defined by equation (\ref{eq:Intensity profile}). The average is taken over all the experiments with the same curved wall parameters $A$, $\lambda$. Intensity, is proportional to bacteria density, and so $I(x)$ is a measure of mean bacteria density in contact with the wall. In figure \ref{experimental profiles}, we show examples of experimental intensity profiles for $\lambda=$ \SI{30}{\micro\meter} and $\lambda= $ \SI{24}{\micro\meter} for three different amplitudes $A$. We can see how, for both wavelengths there is a transition on how the intensity behaves. For low values of $A$, the wall is just slightly curved, and bacteria can move through it easily. The curvature makes bacteria leave the wall, therefore, the intensity values are lower than the flat wall. Also, there is a minimum in $x=0$ corresponding to the valley of the curved wall. This minimum is produced because not all cells that touch the wall will go through the valley. If the contact starts near a peak, the the cell will leave the wall without going through the whole period. Then, we have a critical point, where $I(x)$ looks flat, which can be seen on $A=$ \SI{8.5}{\micro\meter} and $\lambda= $ \SI{30}{\micro\meter}. In this case, bacteria is still moving quickly around the wall, as the intensity is lower than the flat wall. Moreover, the intensity is lower compared to the previous case, because bacteria leave the wall with a higher angle. Finally, if the valley is too narrow, the intensity will increase greatly on the valley as bacteria get trapped on it. The more narrow the valley is, the higher the peak of intensity will be as bacteria get trapped more time.

We call the transition  described as the accumulation transition. The accumulation transition is related to the curvature of the wall, depending on both $A$ and $\lambda$. For example, we can see that for \ref{experimental profiles} a) with $\lambda= $ \SI{30}{\micro\meter} the  critical point is in $A=$ \SI{8.5}{\micro\meter}, but for \ref{experimental profiles} b) $\lambda=$ \SI{24}{\micro\meter} we can not see the behavior described an so the critical point must be between $A=$ \SI{2.8}{\micro\meter} and \SI{5.5}{\micro\meter}. We now present a quantitative description of the accumulation transition.


\subsection{Qualitative predictions}

Qualitatively, the transition can be described as going from fewer bacteria in the valley of the wall to bacteria accumulating in there. This is represented by going from a minimum in $x=0$ to a maximum. To quantify that aspect, we used the fourier coefficent $c_1$ of the function $\cos{(2\pi x/\lambda)}$ calculated by:

\begin{equation}
    c_1 = \frac{2}{\lambda}\int_{-\lambda/2}^{\lambda/2} I(x)\cos\left(\frac{2\pi x}{\lambda} \right)dx = \frac{2}{\lambda} \sum_i I(x_i) \cos\left(\frac{2\pi x_i}{\lambda} \right)\Delta x,
\end{equation}

where $\Delta x$ is the spatial resolution of the profiles and $x_i$ the points in it. Negative values of $c_1$ indicate a minimun on $x=0$ and positive $c_1$ the opposite. The value of $c_1$ depends on the curvature of the wall. 

\afterpage{%
\begin{figure}[H]
	\centering
	\includesvg[width=\linewidth]{imagenes/c1}
	\caption[Coefficient $c_1$ for experiments and simulations]{Color plots of the $c_1$ coefficient in the $A$, $\lambda$ parameter space. The colors are such that $c_1=0$ corresponds to the grey color and extreme negative and positive values are blue and red respectively. Red and blue are not equally spaced since the values for positive $c_1$ are one order of magnitude higher. a) Experimental results for $c_1$, where the number in above each point represents the number $N$ of periods considered in the mean intensity profile $I(x)$. The $A\sim$ \SI{9}{\micro\meter} column has the lowest amount of experiments. Simulation results are in b), c) and d) with their respective parameters. For those three results, $k_{\text{cell}}=0$, meaning no cell interaction was considered. }
	\label{c1 coefficient}
\end{figure}
}



Figure \ref{c1 coefficient} shows color plots of experiments and simulations for $c_1$ in the $A$, $\lambda$ parameter space. The transition from negative to positive values is  seen in gray. In experiments, we can see how the accumulation transition occurs in $A= $ \SI{8.5}{\micro\meter}, $\lambda=$ \SI{30}{\micro\meter} and $A=$ \SI{5.6}{\micro\meter}, $\lambda= $ \SI{27}{\micro\meter}, but for the other wavelengths we lack the resolution in the amplitude to observe the critical point. We only know that is happens between $A\sim$ \SI{3}{\micro\meter} and \SI{5}{\micro\meter}, as discussed previously. In simulations, for now we consider $k_{\text{cell}}=0$ as in the low density regime, bacteria barely interact. We show three different sets of parameters and their results. In figure \ref{c1 coefficient} b) the set of parameters replicates the observed transition with precision. However, $c_1$ values are not exact, as expected. Even experiments performed in different days do not show the exact same profiles, as motility and density vary slightly. Even so, these results should not be viewed in lesser terms. Figures \ref{c1 coefficient} c) and d) support the idea that not all parameters represent the transition trivially. In c), the alignment with the wall is too intense, so bacteria do not get trapped even for higher curvatures. In d) the thermal fluctuations dominate so bacteria can not leave the valleys. We could say that $K$ and $D_r$ have opposite effects. Higher values of $K$ means that bacteria allign with the wall more quickly but higher $D_r$ introduces more thermal noise that make bacteria go to the curved wall. In fact, there is many combinations of $K$ and $D_r$ that replicates the transition. We will call candidates, such pairs of $K$ and $D_r$ that replicate the transition similarly to \ref{c1 coefficient} b). We used the pair $K=3.5$ and $D_r=0.015$ for figure \ref{c1 coefficient} because the values are closer, but as mentioned they are not the same.

\afterpage{%
\begin{figure}[H]
	\centering
	\includesvg[width=\linewidth]{imagenes/candidate_profiles}
	\caption[Comparison of intensity profiles within experiments and candidates]{ Intensity profiles for experiments and candidates. Rows have the same amplitude $A$, and columns the same wavelength $\lambda$. Row order is $A\sim$ \SIlist[list-units=single, list-final-separator = {, }]{3;6;9}{\micro\meter} and column order is $\lambda=$ \SIlist[list-units=single, list-final-separator = {, }]{24;27;30}{\micro\meter}. The x-y axis scales are adjusted depending on the amplitude and wavelength, in pursue of a clear display of the data. }
	\label{candidates intensity profiles}
\end{figure}
}

Figure \ref{candidates intensity profiles} shows intensity profiles obtained in simulations of candidates, compared to the measured in experiments for values of $A$ and $\lambda$ that are of interest for the transition. This comparison is very demanding for the simulations. The intensity profiles in the experiments are subject to a variety of effects that the simulations do not capture. For example, experimental profiles are often asymmetric, possibly due to inhomogeneities in density causing more bacteria to come from one side, in this case the right. But this is not the only important difference. In profiles d), e) we see that the simulations do not predict a drop in the intensity values at the ends of the profile when the transition is formed, only the amplitude of the accumulation is properly predicted by the $K=3.5$, $D_r=0.015$ profile. Also in profile f) the experiment is flat, so $K=5.0$ is the closest curve, in opposition of the previous case. We conclude that there is no perfect candidate to replicate the exact results of all experiments. Since $D_r$ should not depend on curvature is hard to think on a dependence of curvature for the parameters. Nevertheless, the model is close in values and behavior for all of the candidates. The color plots of $c_1$ in $A$, $\lambda$ space are an oversimplification and so they should be interpreted as a qualitative replication of the results in the experiments. Exact quantitative predictions for the profiles are not achievable with this model.

\section{Tracking}

Following methodology described in section 2.2.4, we track bacteria movement. The method determines bacteria position $\textbf{r}_i$ for each frame and then form links between detections to create the trajectories. This is an automatic process subject to errors, but mostly gives correct results. We can calculate the velocity of particle $i$ in a frame $t$ as:

\begin{equation}
    \dot{\textbf{r}}_i(t) =  \frac{\textbf{r}_i(t+dt)-\textbf{r}_i(t)}{dt},
\end{equation}

where is $dt$ is the time difference between two succesive detections in a trajectory. Time resolution is \SI{100}{\milli\second}, but $dt$ can be greater if particles are not detected for a brief time, for example, in the case of collisions. The maximum value of $dt$ allowed is \SI{500}{\milli\second}. In simulations both, $\textbf{r}_i$ and $\dot{\textbf{r}}_i$ are numerically calculated on each time step. Statistics of these vectors, will reveal more information about the dynamics of the system.

\afterpage{%
\begin{figure}[H]
	\centering
	\includesvg[width=\linewidth]{imagenes/speed_distribution}
	\caption[Probability density function of the speed, comparison between non-normalized and normalized speed.]{ Probability density functions for the speed in four different videos. a) and b) distributions for the non-normalized speed $v$ in the bulk and in the curved wall, respectively. c) and d) distributions for the normalized speed $v/v_{\text{bulk}}$ in the same experiments. a) shows distributions in the bulk with different means, but in c) we can see how the distributions in the bulk are comparable thus justifying the usage of the normalized speed instead of $v$. Also, in the curved wall the distributions are similar but the red curved presents an increase on $v=0$. This is important to note, because there is a high variation between experiments.}
	\label{speed distribution: normalization}
\end{figure}
}

\subsection{Speed distribution}

We begin by considering the probability density function of the speed $v$. Caution is required when comparing that quantity for different experiments. Cell motility is not always exactly the same. It is affected by the use of the micropipette, presence of oxygen and centrifugation. In figure \ref{speed distribution: normalization} we show measurments of this probabilities for the experiment with $\lambda=$ \SI{30}{\micro\meter} and $A=$ \SI{5.6}{\micro\meter} in two different days for four videos. Figure \ref{speed distribution: normalization} a) shows the speed on the bulk of the system, namely outside of the bands of both walls. There are major differences on the means of these distributions. In figure \ref{speed distribution} b) the speed distribution on the curved wall is shown. The differences carry on to these probability densities. Considering this, we use the normalized speed $v/v_{\text{bulk}}$ to compare between different experiments, where $v_{\text{bulk}} = \langle v \rangle_{\text{bulk}} $ is the mean speed in the bulk of the system for a specific video. In figures \ref{speed distribution: normalization} c) and d) we plot the results for the normalized velocity. This normalization assures experiments have comparable distributions. From now on, we will only compare the normalized dimensionless velocity, and so experimental distributions will consider all experiments.

\afterpage{%
\begin{figure}[H]
	\centering
	\includesvg[width=\linewidth]{imagenes/speed_distribution_comparison}
	\caption[Comparisons of the probability density functions for the normalized velocity in contact with the curved wall, between experiments and simulations.]{Probability density functions of speed in contact with the curved wall, for experiments and simulations. Same labels as in figure \ref{candidates intensity profiles} and values of $A$, $\lambda$
	}	
	\label{speed distribution: comparison}
\end{figure}
}

In figure \ref{speed distribution: comparison} we compare results from experiments and simulations for the velocity distributions. Comparing with figure \ref{candidates intensity profiles}, is possible to see how the transition corresponds to a decrease of normalized speed in the curved wall to near zero values. \textcolor{red}{Note: Missing simulation results, will elaborate more then.}

\subsection{Velocity profiles}

Following the idea of intensity profiles, we now dedicate to explain velocity profiles. If a particle $i$ is in the curved wall band with position $\textbf{r}_i =(x_i, y_i)$, we can assign a interval in the x-axis defined as $[x_k-\Delta x/2,x_k+\Delta x/2]$ where $x_k$ is the center of the interval where $x_i$ is contained. This creates a set of observations $\mathcal{O}_k$ associated with the interval defined by $x_k$. For each value $x_k$ we define $v(x_k)$ the mean velocity in the set of observations $\mathcal{O}_k$ that are inside the interval. Particles in the band are probably in contact with the wall, so vertical position does not reveal more information. The definition is the same for experiments and simulations, but in experiments $\Delta x $ = \SI{1}{\micro\meter} is considered, meanwhile in simulations $\Delta x $ = \SI{0.32}{\micro\meter} as usual in the intensity profiles. The reason for this change for the experiments is that in the less frequented positions, there is few data, so by losing resolution it is possible to increase the size of $\mathcal{O}_k$ in those cases. This is essential for the cases were bacteria are trapped in the valley. Figure \ref{velocity profiles} h) shows a case where two points have high errorbar due to the small size of $\mathcal{O}_k$. 

\afterpage{%
\begin{figure}[H]
	\centering
	\includesvg[width=\linewidth]{imagenes/velocity_profiles}
	\caption[Velocity profiles compared between experiments and simulations.]{Velocity profiles $v(x)$ for experiments and simulations with the same labels of figure \ref{candidates intensity profiles} and values of $A$, $\lambda$. Errorbars are de $95\%$ interval of confidence por the estimation of the mean $v(x)$. a), b) and c) are low amplitude cases where we can see a reduction on velocity in the valley for experiments, but simulations predict a lower decrease. Apart from that, simulations predict values appropriately, specially the $K=3.5$, $D_r=0.015$ case. In simulations, bacteria slow down because of collision force with the wall.}	
	\label{velocity profiles}
\end{figure}
}

\newpage
% Parrafo del final del caítulo
These observations raise an important question. What is the relevance of a qualitative description? Theoretical studies offer quantitative predictions that inspire new experiments or give a better understanding of the phenomena. In our case, the model is very simple and does not predict intensity profiles exactly. It is hard to trust in the model for predictions in a different scenario. Nevertheless, we think the importance of the model relies precisely on its simplicity. We are able to predict the accumulation transition considering two important physical phenomena, rotational diffusion and the alignment with the wall. The spherocylindrical shape of cells, the friction with the wall, collision between cells, and even hydrodynamic effects caused by the movement of the flagela are not considered in these results. There is so much physics involved in these experiments, but a model with such little considerations predicts the main observed transition. This can mean only one thing, the dynamics of cells near sinusoidal walls is dominated by the effects described in the model. Since rotational difussion is present even for flat walls, we conclude that steric alignment of cells with the wall is the main reason for cells to leave sinusoidal walls. 




\chapter{Conclusions and perspectives}

In this thesis, we studied the swimming of bacteria \textit{Escherichia coli} (\textit{E. coli}) near a sinusoidal boundary. We used low-density suspensions of bacteria to measure their accumulation near the curved surfaces. Through image analysis and bacterial tracking, we measured bacteria's density, speed, and residence times in the region close to the curved wall. We also developed a model of bacteria swimming, considering steric interactions with the surface.

We observed that bacteria movement in the curved surface depends on the parameters of the sinusoidal shape, namely the amplitude $A$ and the wavelength $\lambda$. Their combined effect can be largely summarized in the wall maximum curvature $\kappa=4\pi^2 A /  \lambda^2$. For lower curvatures, bacteria move along the surface quickly, and when bacteria reach a peak, they may leave the surface. As the curvature increases, bacteria become trapped in the valleys and eventually form clusters that last for several seconds. We called this transition the accumulation transition. To characterize this transition, we measured the mean density of bacteria in a period of the sinusoidal wall, through the intensity of the fluorescence of bacteria. We focused on the intensity in the curved wall, which we normalized by the mean intensity in the flat wall. This normalization ensures experiments are comparable. The accumulation transition was characterized via the Fourier coefficient $c_1$ associated with the shape of the intensity profile. We determined that the accumulation transition occurs near a critical curvature $\kappa^* =$ \SI{0.3}{\per \micro \meter}. 

The model considered interactions between bacteria and the wall through the steric alignment intensity $K$ and random reorientations induced by the liquid particles through the rotational diffusion coefficient $D_r$. $K$ and $D_r$ were the only non-fixed parameters of the model. We adjusted the model parameters to replicate the values of $c_1$ resulting in the optimal values $K^*=$ \SI[per-mode = symbol]{3.0}{\radian\per\second} and $D_r^*=$ \SI[per-mode = symbol]{0.015}{\square\radian\per\second}. This values compare well to the reported measurements in the literature. In a flat surface $K=$ \SI[per-mode = symbol]{4.9}{\radian\per\second} was measured \cite{Bianchi20193DInterface}, and a lower value of $K$ is to be expected as bacteria align less with the curved surface. Also $D_r=$ \SI[per-mode = symbol]{0.057}{\square\radian\per\second} was reported for cells swimming far away from boundaries \cite{Drescher2011FluidScattering}. In our experiments, bacteria in the focal swim in contact with a frontal surface, so it is reasonable to have a lower value of $D_r$ in our case due to the geometric restrictions that the surfaces imposes on the liquid and the rotation of the bacteria. 

The optimal values of the model parameters were determined only with $c_1$. Nevertheless, the model replicates other quantities such as mean accumulation in the curved wall, speed profiles near the wall, and contact times with it. This is interesting because the model is very simple. The model does not consider the spherocylindrical shape of cells, the friction with the wall, the collision and alignment between cells, and the hydrodynamic effects caused by the flagella movement. There is plenty of physics involved in these experiments, but a model with minimal ingredients predicts the observed accumulation transition and makes reasonable predictions for all the measurements we made in the experiments. This can mean only one thing, the dynamics of cells near sinusoidal walls is dominated by the effects described in the model. We conclude that the steric alignment of cells with the wall and the rotational diffusion are the primary physical mechanisms that govern the dynamics of bacteria when swimming near a curved surface.

Regarding the mean accumulation of bacteria, we discovered that near the critical curvature, a minimum of the average accumulation along the curved wall is found for the values $A=$ \SI{5.6}{\micro\meter}, $\lambda= $ \SI{27}{\micro\meter}. Nevertheless, an experiment with almost equal curvature but lower amplitude showed a higher accumulation. This is because, for these curvatures, bacteria move around the valley easily, but for higher amplitude, the bacteria leave the wall with a higher inclination and therefore move away from the surface. This is interesting for designing surfaces that reduce biofilm formation because it reveals that curvature is not the only relevant parameter. We believe that semicircular patterns are the best option to reduce accumulation of bacteria, as they are expelled with the highest inclination possible. Nevertheless, our characterization as a function of the curvature is helpful because it is independent of the shape. If we extrapolate the results of this thesis for the semicircular geometery, we can predict an optimal radius $R^*$ around $R^*=(\kappa^*)^{-1}\appox$ \SI{5}{\micro\meter} for semicircular patterns.

Thanks to the measurments of residence time of bacteria near the walls, we determined that the average contact time in the sinusoidal wall is an order of magnitude lower than in the flat surfaces. This decrease could mean that bacteria do not have time to adhere to the surface. In the future, we believe it is necessary to test this surface in more natural situations to measure its effectiveness in preventing biofilm formation. 

The work done in this thesis still has plenty of room for improvement, as was discussed in the previous chapter. For example, we only have experiments with $A\approx$ \SI{9}{\micro\meter} in one day. Due to technical problems with the camera, it was impossible to generate more experimental data. We are also lacking resolution in the amplitudes, because for the wavelengths \SIlist{21;24}{\micro\meter} we cannot observe the accumulation transition optimally. Increasing the number of experiments in this range of amplitudes is crucial to elucidate the differences observed in this range. Therefore an increase of the resolution in the  $(\lambda, \ A)$ space will better describe the accumulation transition. Moreover, the experiments will benefit from data from lower amplitudes as it is expected that for curvatures near zero the behavior in the curved wall will recover the behavior of the flat surface. We also believe that a tracking-based bacterial density measure will allow a more direct comparison between experiments and simulations, but it was not considered due to time restrictions.

In summary, this thesis contributes to understanding how shape alters the accumulation of bacteria near a surface. We observe a transition in the accumulation depending on the curvature in the wall. We explain this phenomenon by a simple model that considers an alignment with the wall and rotational diffusion. By adjusting the model parameters to replicate the transition, the dependence of bacterial speed and mean density with respect to the curvature is replicated, indicating that the system's dynamics are correctly represented. This implies that the steric alignment dominates other effects typically present in experiments but not included in the model. In addition, we have shown that these surfaces reduce the accumulation of bacteria on the wall considerably as the bacteria are in contact for times an order of magnitude shorter than near flat walls. This study promises that control of biofilm formation by optimization of surface curvature is possible.



\bibliographystyle{plain}
\bibliography{references}

\begin{appendices}
\chapter{Normalized intensity and speed profiles for all cases}

\vspace{-30pt}
\begin{figure}[ht]
	\centering
	\includesvg[width=\linewidth]{anexoA/Curvature = 0.127}
	\includesvg[width=\linewidth]{anexoA/Curvature = 0.157}
\end{figure}
\newpage
\begin{figure}[ht]\ContinuedFloat
	\centering
	\includesvg[width=\linewidth]{anexoA/Curvature = 0.192}
	\includesvg[width=\linewidth]{anexoA/Curvature = 0.242}
	\includesvg[width=\linewidth]{anexoA/Curvature = 0.250}    
\end{figure}
\newpage
\begin{figure}[ht]\ContinuedFloat
	\centering 
	\includesvg[width=\linewidth]{anexoA/Curvature = 0.303}
	\includesvg[width=\linewidth]{anexoA/Curvature = 0.373}
	\includesvg[width=\linewidth]{anexoA/Curvature = 0.377}    
\end{figure}
\newpage
\begin{figure}[ht]\ContinuedFloat
	\centering
	\includesvg[width=\linewidth]{anexoA/Curvature = 0.455}
	\includesvg[width=\linewidth]{anexoA/Curvature = 0.474}
	\includesvg[width=\linewidth]{anexoA/Curvature = 0.487}    
\end{figure}
\newpage
\begin{figure}[ht]\ContinuedFloat
	\centering
	\includesvg[width=\linewidth]{anexoA/Curvature = 0.562}
	\includesvg[width=\linewidth]{anexoA/Curvature = 0.590}
	\includesvg[width=\linewidth]{anexoA/Curvature = 0.707}    
\end{figure}
\newpage

\begin{figure}[ht]\ContinuedFloat
	\centering
	\includesvg[width=\linewidth]{anexoA/Curvature = 0.720}
	\includesvg[width=\linewidth]{anexoA/Curvature = 0.895}
    \caption[Normalized intensity and speed profiles for all amplitudes and wavelengths for the experiments and the simulation with the pair of optimal parameters.]{Normalized intensity and speed profiles for all amplitudes and wavelengths for the experiments and the simulation with the pair of optimal parameters. All profiles are labeled with the amplitude $A$, wavelength $\lamba$ and the associated maximum curvature $\kappa=4\pi^2A/\lambda^2$. The figures are sorted by the curvatures. }    
\end{figure}
\label{appendix:figures}
\end{appendices}
\end{document}