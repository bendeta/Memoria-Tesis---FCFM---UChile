\chapter{Models and simulations}

\section{Theoretical framework}

This section explains details of the model considered and the criteria used for its election. We consider an agent-based model of spherical active particles with overdamped dynamics in a 2-dimensional representation. 

\subsection{Agent-based models} 

In biophysics, models can be sorted into two classes, agent-based, and continuum models. Agent-based models simulate the dynamics of cells individually, offering accurate system descriptions. Depending on the complexity of the model, they can be costly in a computational sense. On the other hand, continuum models consider a coarse-grained description with density and velocity fields. We work in a low-density regime, so a continuum description is meaningless. Therefore, we will use agent-based models. For simplicity, we consider cells with spherical shapes. In reality, the cell shape depends on the growth phase of the bacteria culture, but \textit{E.coli}'s body is a spherocylinder with typical dimensions of \SI{0.5}{\micro\meter} in radius and \SI{2}{\micro\meter} in length.

\subsection{Overdamped dynamics}

To describe the dynamics of single cells, we invoke Newton's second law: 

\begin{equation}
	m\ddot{\textbf{r}} = \sum \textbf{F} = \textbf{F}_{\text{hydro}} + \textbf{F}_{\text{flag}} + \textbf{F}_{\text{brown}} + \textbf{F}_{\text{coll}} ,
\end{equation}

where we define four different force sources, $\textbf{F}_{\text{hydro}}$ the hydrodynamic friction produced by the fluid, $\textbf{F}_{\text{flag}}$ the force produced by the flagella, $\textbf{F}_{\text{brown}}$ the Brownian force due to collisions with the liquid molecules, and $\textbf{F}_{\text{coll}}$ produced due to cell-cell and cell-surface collisions. We can deduce an important fact widely accepted in biophysics with the following analysis. To analyze the role of inertia, let's imagine we have isolated \textit{E.coli} bacterium that suddenly stops swimming, so $\textbf{F}_{\text{flag}} = \textbf{F}_{\text{coll}} =0$. Although the influence of $\textbf{F}_{\text{brown}}$ is essential, its effect is stochastic. If we repeat the situation many times and average the ensemble, its contribution is expected to be zero. Therefore, for the moment, we will neglect it. Then we are left with the equation:

\begin{equation}
	m\ddot{\textbf{r}} = \textbf{F}_{\text{hydro}} = -\gamma \dot{\textbf{r}} = - 6 \pi R \eta \dot{\textbf{r}},
\end{equation}

where we used Stokes' law to calculate the drag force coefficient $\gamma$ with $R=$ \SI{0.5}{\micro\meter} being the particle radius and $\eta=$ \SI{d-3}{\pascal\cdot\second} the viscosity. Since the mass of an \textit{E.coli} is $m=$ \SI{d-12}{\gram}, the characteristic time of slowing down is given by $ m/(6 \pi R \eta) \approx $ \SI{d-7}{\second}. This means that in the timescale of \SI{1}{\micro\second} the bacteria should have stopped. We record at ten fps in our experiments, so the detention is immediate for our time resolution. This means that we are working on the overdamped limit, where inertia is negligible, and so it is correct to simplify the dynamics as:

\begin{equation} \label{eq:overdamped_model}
	\gamma \dot{\textbf{r}} = \textbf{F}_{\text{flag}} + \textbf{F}_{\text{brown}} + \textbf{F}_{\text{coll}}.
\end{equation}

\subsection{Rotational diffusion}

We considered two different effects of the fluid in the bacterial dynamics. First, $\textbf{F}_{\text{hydro}}$ is a phenomenological dynamical drag force that represents the average effect of the liquid on an object moving through the fluid, and second, $\textbf{F}_{\text{brown}}$ of stochastic kind accounting for the thermal fluctuations. The latter produces a mean square displacement of $\langle r^2 \rangle (t) = 4 D t $ where $D$ is the diffusion constant given by the fluctuation-dissipation theorem \cite{Soto2016KineticPhenomena}:

\begin{equation}
	D = \frac{k_bT}{\gamma} \approx \SI[per-mode = symbol]{1.5d-1}{\square\micro\meter \per \second},
\end{equation}
 
where the same value for $\gamma$ was used. This diffusion constant is purely translational. Nevertheless, the collisions between cells and water molecules also exert torques on the bacteria, meaning that the orientation of swimming is subject to an equivalent stochastic process. This process has a rotational diffusion constant $D_r$ independent of $D$. For smooth-swimming \textit{E.coli} it has been measured at $D_r=$ \SI[per-mode = symbol]{0.057}{\square\radian \per \second} \cite{Drescher2011FluidScattering}. We can perceive the importance of rotational diffusion by neglecting collisions in equation \eqref{eq:overdamped_model} and calculating the mean square displacement as described in\cite{Lauga2020TheMotility}.

\begin{equation} \label{eq:flagellar force}
	 \dot{\textbf{r}} =\frac{\textbf{F}_{\text{flag}} + \textbf{F}_{\text{brown}} }{\gamma} = u \textbf{p} + \textbf{F}^*_{\text{brown}},
\end{equation}

where $u=$\SI[per-mode = symbol]{20}{\micro\meter \per \second} is the mean speed of bacteria, $\textbf{p}$ the previously defined vector of orientation, and $\textbf{F}^*_{\text{brown}}$ is just the force with the coefficient $\gamma$ absorbed. Integrating equation \eqref{eq:flagellar force} in time we obtain:

\begin{equation} \label{eq:positon}
	\textbf{r}(t) - \cancelto{0}{\textbf{r}(0)}  = \int_0^t [u \textbf{p}(t^\prime) +  \textbf{F}^*_{\text{brown}}(t^\prime)  ]dt^\prime ,
\end{equation}

Taking the dot product of \eqref{eq:flagellar force} and \eqref{eq:positon} we obtain an equation for the time derivative of the square displacement.

\begin{equation} \label{eq:derivative of msd}
	\dot{\textbf{r}} \cdot \textbf{r} = \frac{1}{2}\frac{d}{dt} (r^2) = u^2 \int_0^t [ \textbf{p}(t^\prime) \cdot \textbf{p}(t) + \textbf{F}^*_{\text{brown}}(t) \cdot \textbf{F}^*_{\text{brown}}(t^\prime)]dt^\prime + \text{irrelevant terms}.
\end{equation}


All the terms in the right side of equation \eqref{eq:derivative of msd} are stochastic and vary from cell to cell. Nevertheless, their average for all particles has properties that allow progress in the calculation. For example, the terms cataloged as irrelevant consider dot products of two quantities completely uncorrelated, the director vector $\textbf{p}$ and the force $\textbf{F}^*_{\text{brown}}$. Therefore after averaging on the ensemble of cells, the result is zero, and we are left with: 

\begin{equation} 
	\frac{d}{dt} \langle r^2 \rangle &= 2u^2 \int_0^t \langle \textbf{p}(t^\prime) \cdot \textbf{p}(t) + \textbf{F}^*_{\text{brown}}(t) \cdot \textbf{F}^*_{\text{brown}}(t^\prime) \rangle dt^\prime.
\end{equation}


The quantity $\langle\textbf{p}(t^\prime) \cdot \textbf{p}(t)\rangle$ is the mean time correlation of the vector director \textbf{p} and depends on the rotational diffusion coefficient as $e^{-2D_r|t-t^\prime|}$ \cite{Lauga2020TheMotility}. On the other side $\textbf{F}^*_{\text{brown}}$ is assumed to be an uncorrelated noise, that satisfies $\langle \textbf{F}^*_{\text{brown}}(t) \cdot \textbf{F}^*_{\text{brown}}(t^\prime) \rangle = 2D\delta(t^\prime -t)$. Therefore, integrating over both $t$ and $t^\prime$ we obtain:


\begin{equation} 
    \langle r^2 \rangle (t) = \frac{u^2}{D_r}\left( t + \frac{e^{-2D_rt}}{2D_r} - \frac{1}{2D_r} \right) + 4Dt
\end{equation}

At short times,  $t \ll D_r^{-1} =$ \SI{18}{\second} the exponential can be expanded in a Taylor series $e^{-2D_rt}\approx 1 - 2D_rt + 2(D_r t^2) + \mathcal{O}(t^3)$ giving $\langle r^2 \rangle (t)  \approx (ut)^2$, which means that at short times particles swim is straight lines. Bacteria swimming is more important than diffusion even at our lowest time scale of $t=$ \SI{0.1}{\second}, as in that case $4Dt$ is two order of magnitude lower than $(ut)^2$. In the other case $t \gg D_r^{-1}$ the mean square displacement is $(u^2/D_r+4D)t$. Adding the translational diffusion discussed previously, we deduce an effective diffusion constant for long times $D_{\text{eff}}$ given by:

\begin{equation}
    D_{\text{eff}} = D + \frac{u^2}{4D_r}
\end{equation}

For the typical values mentioned, $D_{\text{eff}}\approx$ \SI[per-mode = symbol]{2d3}{\square\micro\meter \per \second} which is four orders of magnitude larger than $D$. This means that bacteria swimming makes translational Brownian motion irrelevant compared to the effects of rotational diffusion. We conclude that $\textbf{F}_{\text{brown}}$ can be set to zero for simplicity without losing relevant dynamics. We are left with the equation:

\begin{equation} \label{eq:final_model}
    \dot{\textbf{r}} = u\textbf{p} + \frac{1}{\gamma}\textbf{F}_{\text{coll}} = u\textbf{p} + \textbf{F}^*_{\text{coll}} ,
\end{equation}

where $\textbf{F}^*_{\text{coll}}$ absorbs the drag coefficient $\gamma$ and hence has units of speed.

\subsection{Alignment with the wall}
\label{section:steric alignment}

We observe that bacteria interacting with the curved and flat walls, suffer a torque that aligns them with the wall \cite{Bianchi2017HolographicBacteria}. This alignment has its origin on steric forces. If we consider equation \eqref{eq:final_model} and take the dot product with the vector perpendicular to the surface  $\hat{\textbf{n}}_w$ at the point of contact, we obtain:

\begin{equation} \label{eq:steric_force}
   \cancelto{0}{\dot{\textbf{r}} \cdot \hat{\textbf{n}}_w}  = u\textbf{p} \cdot \hat{\textbf{n}}_w + \textbf{F}^*_{\text{coll}} \cdot \hat{\textbf{n}}_w ,
\end{equation}

where $\dot{\textbf{r}} \cdot \hat{\textbf{n}}_w = 0$ is imposed as bacteria do not enter the surface. Since $\textbf{F}^*_{\text{coll}} \cdot \hat{\textbf{n}}_w $ is exclusively perpendicular to the wall, equation \eqref{eq:steric_force} gives the magnitude of such force. Then we can write the equation for the angle of swimming $\theta$ considering that the inertia of the cell is negligible and so that the sum of torques must be zero. Simplifying the system, we consider the torques repect to the center of mass as the rotational drag and the torque exerted by the wall. Other torques can be considered for more complete models to explain observations such as circular trajectories and longer residence times in the wall \cite{Lauga2006SwimmingBoundaries, Sipos2015HydrodynamicWalls}. For this thesis we decided to work with a simple model because these effects are not required to explain the abandonment of the curved wall by bacteria. Utilizing equation \eqref{eq:steric_force}, 

\begin{equation} \label{eq:deduction_of_K}
    0 = - \gamma_r \dot{\theta} \hat{\textbf{z}} - u \gamma (\textbf{p} \cdot \hat{\textbf{n}}_w) (\textbf{r}_{cm} \times \hat{\textbf{n}}_w),
\end{equation}

where $\gamma_r$ is the rotational drag coefficient and $\textbf{r}_{cm}$ the vector from the center of mass to the point of contact, as we are calculating the torque with respect to the center of mass. Therefore, $\textbf{r}_{cm} \times \hat{\textbf{n}}_w = L_{cm} (\textbf{p} \cdot \hat{\textbf{t}}_w) \hat{\textbf{z}}$ where $\hat{\textbf{t}}_w$ is the tangential vector to the wall in the point of contact that satisfies $\hat{\textbf{z}} = \hat{\textbf{t}}_w \times \hat{\textbf{n}}_w$  and we used $\textbf{r}_{cm} = L_{cm} \textbf{p} $ to take into consideration the contribution of the flagella to the center of mass. We are left with the equations:

\begin{align}
    \textbf{p} &= \cos{\theta}\hat{\textbf{x}}+\sin{\theta}\hat{\textbf{y}}, \\
    \dot{\theta} &= -K (\textbf{p} \cdot \hat{\textbf{t}}_w)  (\textbf{p} \cdot \hat{\textbf{n}}_w) \Gamma(\textbf{r}, \textbf{p}),
    \label{eq:wall alignment}
\end{align}

where $K\equiv u L_{cm}\gamma / \gamma_r$. The scalar $K$ controls the intensity of the alignment and in \cite{Bianchi20193DInterface} a fit of experimental data yields $K=$ \SI[per-mode = symbol]{4.9}{\radian \per \second}. In our study we consider $K$ as a free parameter with values between \SIlist[per-mode = symbol, list-units=single]{0;7}{\radian\per\second}. This is because we are considering the more complicated case of a curved wall. Also, $\Gamma(\textbf{r}, \textbf{p})$ is a step function that indicates if the bacteria is in contact with the wall or not. It is equal to $1$ when the distance from the center of the particle $\textbf{r}$ to the closest point of the wall is smaller than one radius, and its swimming direction points towards the wall i.e. $\textbf{p} \cdot \hat{\textbf{n}}_w < 0$ indicating the cell is going into the wall and not away from it. Otherwise $\Gamma$ equals zero.  Equation \eqref{eq:wall alignment} will align the vector $\textbf{p}$ with the $\pm\hat{\textbf{t}}_w$ depending on the sign of $\textbf{p} \cdot \hat{\textbf{t}}_w$. This effect is only considered for the flat and curved wall as the frontal wall is already taken into account because simulations are two-dimensional.

\begin{wrapfigure}{r}{0.5\linewidth}
\centering
\includesvg[width=\linewidth,height=4cm]{imagenes/cp_wall_diagram}
\caption[Diagram of the components involve in the wall allignment]{Diagram of the components involve in the wall alignment. $O$ represents the origin of coordinates.}
\label{circular trajectories}
\end{wrapfigure}

We calculate the point of contact as the closest point in the walls to the cell by dividing the interval $[x-R, x+R]$ in 300 points where $x = \textbf{r} \cdot \hat{x}$. Then, we calculate the distance between the cell and the position obtained with the parametric definition of the wall for each point. The point $\textbf{r}_w$ with the lowest distance is the closest with a precision of \SI{3d-3}{\micro\meter}. This ``brute force" algorithm works because a point outside of the interval $[x-R, x+R]$ is not in contact with the cell, and it is convenient because the distance to the wall as a function of the x-axis has many local minima.

% \vspace{1cm}

\section{Simulations}

Equations \eqref{eq:final_model}--\eqref{eq:wall alignment} summarize the physics involved in the description of the system. We are only missing two aspects; rotational diffusion and the details of $\textbf{F}_{\text{coll}}^*$. This section first explains how we implement these phenomena, summarize the model, and finally explain how the simulations are performed.
 
\subsection{The collision force}

$\textbf{F}^*_{\text{coll}}$ is the force produced by the collision of cells with other cells or the wall, but divided by the drag coefficient $\gamma$. We considered it as an elastic force for both cell-cell and cell-wall collisions. We parameterize these forces through the following equation: 

\begin{equation}
   \textbf{F}^*_{\text{coll}} &= \sum_{w} k_{\text{wall}} (R-d_{iw}) \hat{\textbf{d}}_{iw}\Theta(R-d_{iw}) + \sum_{j \in \mathbb{NN}} k_{\text{cell}} (2R-d_{ij}) \hat{\textbf{d}}_{ij}\Theta(2R-d_{ij}),
\end{equation}
 
where $d_{iw}$, $\hat{\textbf{d}}_{iw}$ represent the distance and the unit vector between the particle $i$ and the closest point of the wall $w$, and $d_{ij}$, $\hat{\textbf{d}}_{ij}$ are the same but between the particles $i,j$ where $j$ is one of the particles in the nearest neighbor of $i$ set $\mathbb{NN}$. $\Theta$ is the usual Heaviside function. Finally the parameters $k_{\text{wall}}$, $k_{\text{cell}}$ are the intensities of these forces, and have units of \SI[per-mode = symbol]{}{\micro\meter\per\second\per\meter}. We consider $k_{\text{wall}}=$ \SI[per-mode = symbol]{1d3}{\micro\meter\per\second\per\meter}, meanwhile $k_{\text{cell}}$ will be considered as zero and the increased to study the effects of cell in the phenomena observed.

\subsection{Rotational diffusion}

 Rotational diffusion is naturally added to \eqref{eq:wall alignment} as a Gaussian white noise $\eta_i(t)$ \cite{Digregorio2018FullSeparation,Caporusso2020Motility-InducedSystem} meaning it has zero mean and its completly uncorrelated in time and between particles, $\langle \eta_i(t)\eta_j(t^\prime)  \rangle \propto \delta_{ij}\delta (t-t^\prime)$. This white noise will change randomly the direction of swimming $\textbf{p}$ of the cell. We will go into more detail in this section.
 
\subsection{Final model}

The final set of equations that are used in the model are:

\begin{align}
    \label{eq:dynamics of position}
    \dot{\textbf{r}}_i &= u\textbf{p}_i + \textbf{F}^*_{\text{coll}}, \\
    \textbf{p}_i &= \cos{\theta_i}\hat{\textbf{x}}+\sin{\theta_i}\hat{\textbf{y}}, \\
    \label{eq:dynamics of angle}
    \dot{\theta}_i &= -K (\textbf{p}_i \cdot \hat{\textbf{t}}_w)  (\textbf{p}_i \cdot \hat{\textbf{n}}_w) \Gamma(\textbf{r}_i, \textbf{p}_i) + \eta_i(t), \\
    \label{eq:elastic force}
    \textbf{F}^*_{\text{coll}} &=   k_{\text{wall}} (R-d_{iw}) \hat{\textbf{d}}_{iw}\Theta(R-d_{iw}) + \sum_{j \in \mathbb{NN}} k_{\text{cell}} (2R-d_{ij}) \hat{\textbf{d}}_{ij}\Theta(2R-d_{ij}).
\end{align}

All of the quantities used in these equations and on the model are described in table \ref{table:model parameters}. These equations are written for the dynamics of a circular particle $i$, involving other entities such as the wall and the nearest neighbors. Equations \eqref{eq:dynamics of position}--\eqref{eq:elastic force} belong to the class of Langevin equations due to their stochastic nature.

% Tabla del modelo
\begin{table}[!h]
   \centering
    \small
    \caption[Summary of the quantities used in the simulations]{Quantities used for the model, with their description and values if adequate.For parameters that take a range of values, that column will have a $-$ symbol and in Chapter 4, results will have that value specified. }
    \begin{tabularx}{\textwidth}{lXl}
    \hline\noalign{\smallskip}
         Variable  & Description & \quad   \\
    \noalign{\smallskip}\hline\noalign{\smallskip}
         \textbf{r} & Position of the particle $i$. & \quad \\ 
         \textbf{p} & Direction of swimming of the particle $i$, defined by the angle $\theta_i$. & \quad \\
         $\eta$ & White noise associated with rotational diffusion. & \quad \\
         $\hat{\textbf{t}}_w$ & Unitary vector tangent to the wall in the closest point to the cell $i$. & \quad \\
         $\hat{\textbf{n}}_{w}$ & Unitary vector normal to the wall in the closest point to the cell $i$. The normal vectors points away from the wall. & \quad \\
         $\hat{\textbf{d}}_{iw}$ & Unitary vector pointing from the closest point in the wall to the cell $i$. & \quad \\
         $d_{iw}$ & Distance between the cell $i$ and the closest point of the wall. & \quad \\
         $\mathbb{NN}$ & Set of nearest-neighbors of the particle $i$. & \quad \\
         $\hat{\textbf{d}}_{ij}$ & Unitary vector pointing from the neighbor $j$ to the cell $i$. & \quad \\
         $d_{ij}$ & Distance between the cell $i$ and the neighbor $j$. & \quad \\
    \hline\noalign{\smallskip}
        Parameter  & Description & Value   \\
    \noalign{\smallskip}\hline\noalign{\smallskip}
        $\rho$ & Density of cells. This density relates to number of particles seen in the focal plane. & \SI[per-mode = symbol]{3d-3}{\cells \per \square\micro\meter} \\
        $u$ & Swimming speed. & \SI[per-mode = symbol]{20}{\micro\meter\per\second} \\
        $R$ & Radius of the cells. & \SI{0.5}{\micro\meter} \\
         $K$ & Magnitude of the aligment with the wall. & - \\ 
         $D_r$ & Rotational diffusion coefficient. & - \\ 
         $k_{\text{wall}}$ & Elastic constant for cell-wall collisions. & \SI[per-mode = symbol]{1d3}{\micro\meter\per\second\per\meter}\\ 
         $k_{\text{cell}}$ & Elastic constant for cell-cell collisions. & - \\
         $\Delta t$ & Time step for the integration of the equations. & \SI{d-3}{\second} \\
         $\Delta t_r$ & Time step for the recording of data. & \SI{d-1}{\second} \\
         $T$ & Time duration of the simulations. & \SI{200}{\second} \\
    \hline\noalign{\smallskip}
    \end{tabularx}
    \label{table:model parameters}
\end{table}

Quantities with the lower index $w$ depend on the wall. Simulations have a flat and a curved wall with amplitude $A$ and wavelength $\lambda$. The parametric definition of the walls positions are $y=y_f$ for the flat and $y=y_c+A\sin{\frac{2\pi x}{\lambda}}$ for the curved wall. The values of $y_f, y_c$ are so that the mean distance between walls is \SI{100}{\micro\meter}. Also, periodic boundary conditions will be applied. To avoid problems with discontinuities in the wall, the length of the channel is adjusted to have an integer number of wavelengths and be greater than \SI{300}{\micro\meter}. The total particle number is adjusted so that all simulations have the same particle density $\rho$. These values are similar to the experiments.

\subsection{Numerical integration}

Typically, there are many options to solve a differential equation numerically. In this case, the equations are somewhat simple, so it is tempting to integrate \eqref{eq:dynamics of position} and \eqref{eq:dynamics of angle} with Euler's method. That integration method is a first-order method which means that during a step $\Delta t$ of integration, the integrands are taken as constants. Therefore we would obtain:

\begin{align}
    \textbf{r}_i(t+\Delta t) &=  \textbf{r}_i(t) + \Delta t[u \textbf{p}_i(t) + \textbf{F}^*_{\text{coll}}(t)], \\
    \label{eq:wrong integration}
    \theta_i(t+\Delta t) &=  \theta_i(t) + \Delta t[- K (\textbf{p}_i(t) \cdot \hat{\textbf{t}}_w)  (\textbf{p}_i(t) \cdot \hat{\textbf{n}}_w) \Gamma(\textbf{r}_i(t), \textbf{p}_i(t)) + \eta_i(t)].
\end{align}

However, there is a problem with the last term of equation \eqref{eq:wrong integration}, which relies on the timescale in which $\eta$ acts. Remember that $\eta$ represents the torque produced by collisions with the liquid molecules. These thermal fluctuations have a characteristic time between collisions of $\sim$\SI{1.9d-13}{\second} \cite{Soto2016KineticPhenomena}. That means we have two options, either we consider a minimal time step $\Delta t$ so the hypothesis that $\eta$ is constant in the interval is true, or we treat $\eta$ separately. The latter is the best option, as $\eta$'s timescale is much smaller than all the others present in the system. The correct equations are:

\begin{align}
    \textbf{r}_i(t+\Delta t) &=  \textbf{r}_i(t) + \Delta t[u \textbf{p}_i(t) + \textbf{F}^*_{\text{coll}}(t)], \\
    \theta_i(t+\Delta t) &=  \theta_i(t) - K \Delta t  (\textbf{p}_i(t) \cdot \hat{\textbf{t}}_w)  (\textbf{p}_i(t) \cdot \hat{\textbf{n}}_w) \Gamma(\textbf{r}_i(t), \textbf{p}_i(t)) \nonumber \\
     &\ \ \ + \int_t^{t+\Delta t}\eta_i(t^\prime)dt^\prime .
\end{align}


Then a new problem arises, how do we calculate the integral of $\eta$. We can find the answer on stochastic calculus, and here we describe one of the possible demonstrations. First, we treat $\eta$ as what it is, a discrete function representing all the collisions with liquid molecules.

\begin{equation}
    \int_t^{t+\Delta t}\eta(t^\prime)dt^\prime = \sum_{n=1}^N \phi\eta_n,
\end{equation}

where $\eta_n$ is proportional to the angle displacement produced in a certain collision $n$. The magnitude of these rotations is contained in $\phi$ so that we can treat them as standard normal Gaussians. As we stated, these collision displacements are also uncorrelated. Therefore, we can use that the distribution of the sum of two gaussian uncorrelated variables, $x\sim\mathcal{N}(\mu_x,\sigma_x^2)$ and $y\sim\mathcal{N}(\mu_y,\sigma_y^2)$, is given by $\mathcal{N}(\mu_x+\mu_y,\sigma_x^2+\sigma_y^2)$ and by induction obtain:

\begin{equation}
    \sum_{n=1}^N \eta_n \sim \mathcal{N}(0,N) = \sqrt{N}\mathcal{N}(0,1),
\end{equation}

were the last equality is in the sense of probability density distribution. Then, the total number of collisions $N$ is proportional to $\Delta t$, for instance $N=\alpha \Delta t$, where $\alpha$ is a rate of collisions. Then we can write:

\begin{align}
    \textbf{r}_i(t+\Delta t) &=  \textbf{r}_i(t) + \Delta t[u \textbf{p}_i(t) + \textbf{F}^*_{\text{coll}}(t)], \\
    \label{eq:correct integration}
    \theta_i(t+\Delta t) &=  \theta_i(t) - K\Delta t (\textbf{p}_i(t) \cdot \hat{\textbf{t}}_w)  (\textbf{p}_i(t) \cdot \hat{\textbf{n}}_w) \Gamma(\textbf{r}_i(t), \textbf{p}_i(t)) \nonumber \\
     &\ \ \ + \sqrt{\phi^2\alpha\Delta t} \eta(t).
\end{align}

Here $\phi$ is in \SI{}{\radian} and $\alpha$ in \SI{}{\per\second} so $\phi^2\alpha$ has units of \SI{}{\square\radian\per\second}, the same as the rotational diffusion coefficient. Therefore, we naturally obtained the diffusion coefficient, normally defined as $2D_r\equiv\phi^2 \alpha$. As already mentioned, for smooth swimmer \textit{E.coli} in liquids it has been measured that $D_r=$ \SI[per-mode = symbol]{0.057}{\square\radian \per \second}. Nevertheless we consider $D_r$ as a free parameter with values between $D_r=$ \SIrange[per-mode = symbol, range-units=single
]{0.01}{0.05}{\square\radian \per \second} as is expected to be lower in our case, due to the constraints of the walls. Therefore $K$ and $D_r$ are the free parameters of the model.

The result of equation \eqref{eq:correct integration} is really important for the consistency of the numerical integration, since the equation $\langle \eta(t)\eta(t^\prime)  \rangle &= \delta (t-t^\prime)$ implies that $\eta$ has units of \SI{}{\second^{-1/2}} so multiplying by $\Delta t$ instead of $\sqrt{\Delta t}$ is dimensionally wrong.


\subsection{Intensity profiles in simulations}

In experiments we construct intensity profiles using the intensity of bacteria fluorescence. In simulations there is no such thing, we can only measure bacteria positions $\textbf{r}_i$. Therefore, we have to measure the mean bacteria density $n(x)$. For each particle in the band of a wall we can assign a interval in the x-axis defined as $[x-\Delta x/2,x+\Delta x/2]$ where $x_i = \textbf{r}_i \cdot \hat{x} - x_{\text{valley}}$ is contained. Here $x_{\text{valley}}$ is the position of the nearest valley to the particle, meaning $x$ is on a interval of one wavelength $\lambda$. The resolution $\Delta x = $ \SI{0.32}{\micro\meter} is the same as in the experiments. Mean bacteria density $n(x)$ is just the count of bacteria that were in an interval defined by its center $x$. 

Obviously $n(x)$ is not the same as intensity profiles. To create a comparable quantity, we convolve $n(x)$ with a Gaussian function $G(x)=\exp(-x^2/(2R))$. We call the result the intensity profile of the simulation:

\begin{equation} \label{eq:Intensity profiles in simulations}
    I(x) = n(x) * G(x)
\end{equation}

The amplitude of the Gaussian does not matter as we will normalize by the mean accumulation in the flat wall. This treatment means we consider particles having a Gaussian intensity in space. This is not equal to the intensities measured in experiments because bacteria are not spherical. Nevertheless, treating intensity not spherically in simulations would be incorrect, leading to inconsistencies.

\subsection{Algorithm}

The algorithm used in the simulations is described in the following steps.

\begin{itemize}
    \item[1.] Create a random initial condition for all the particles positions and swimming directions. If a particle is out of boundaries or in contact with a wall, its initial conditions are generated again. Then start iterating the time steps.
    \item[2.] Update the closest point of the wall for all particles close to the walls.
    \item[3.] Integrate the equation for $\theta_i$.
    \item[4.] Determine the $\mathbb{NN}$ set for all particles.
    \item[5.] Calculate the force $\textbf{F}^*_{\text{coll}}$.
    \item[6.] Integrate the equation for $\textbf{r}_i$.
    \item[7.] Every time interval $\Delta t_r$ record the relevant data. This avoids high time correlations between measurements.
    \item[8.] Stop when a time $T$ has passed.
\end{itemize}

The algorithm was implemented in a c++ program with object oriented programming. In figure \ref{sim trajectories} we display trajectories for the parameters $K=$ \SI[per-mode = symbol]{5}{\radian\per\second}, $D_r=$ \SI[per-mode = symbol]{0.05}{\radian\per\second} and $k_{\text{cell}}=0$.

\begin{figure}
	\centering
	\includesvg[scale=1]{imagenes/sim_trajectories}
	\caption[Example of trajectories for the simulations]{Example of trajectories  obtained with the simulations for the parameters $K=$ \SI[per-mode = symbol]{5}{\radian\per\second}, $D_r=$ \SI[per-mode = symbol]{0.05}{\square\radian\per\second} and $k_{\text{cell}}=$ \SI[per-mode = symbol]{0}{\micro\meter\per\second\per\meter}. The star represents the start of each track and the triangles the end of them. The values of $A$ and $\lambda$ are included on the top of the plot. Trajectories last for \SI{10}{\second}.}
	\label{sim trajectories}
\end{figure}
