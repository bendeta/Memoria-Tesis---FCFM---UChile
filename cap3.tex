\chapter{Models and simulations}

\section{Theoretical framework}

We consider an agent-based model of spherical active particles with overdamped dynamics in a 2-dimensional representation. This section explains what it means and the reasons for these model features.


\subsection{Continuum vs agent-based models} 

In our case, we are interested in modeling an active suspension. Active suspensions exhibit different phenomena depending on the bacteria and the environment. Thus, the model's features vary depending on the specific properties of the case of study. 

It is customary to construct microscopic models that focus on the swimming of individual bacterium. These are called agent-based models. The dynamics will consist of many terms of clear origin, allowing for a direct interpretation. 

There is also an alternative modeling approach based on deriving a conservation equation for the distribution of cells. This continuum approach considers that there is a distribution $\Psi(\textbf{r},\textbf{p},t)$ that models the probability of finding a particle at position $\textbf{r}$ swimming in the direction $\textbf{p}$ at time t. The evolution for $\Psi(\textbf{r},\textbf{p},t)$ is described by a continuity equation

\begin{equation}
\partial_t \Psi + \nabla_{\textbf{r}} \cdot ( \dot{\textbf{r}} \Psi ) + \nabla_{\textbf{p}} \cdot ( \dot{\textbf{p}} \Psi   ) = 0,
\end{equation}

where the quantites $\dot{\textbf{r}}$ and $\dot{\textbf{p}}$ are often obtained from microscopic models. The $\Psi(\textbf{r},\textbf{p},t)$ distribution allows to calculate statistical fields such as bacteria concentration $c(\textbf{r},t)$ and mean swimming orientation $\textbf{n}(\textbf{r},t)$ as integrals on $\textbf{p}$ of $\Psi$:

\begin{align}
	c(\textbf{r},t) &= \int \Psi(\textbf{r},\textbf{p},t) d\textbf{p}, \\
	\textbf{n}(\textbf{r},t) &= \int \textbf{p} \Psi(\textbf{r},\textbf{p},t)d\textbf{p}.
\end{align}

The field dynamics are derived by integrating the continuity equation for $\Psi$. This coarse-grained description of the system is enough to understand many phenomena such as pattern formation and flocking \cite{Saintillan2008InstabilitiesSimulations,Toner1998FlocksFlocking}. This procedure is widespread in the biophysics community because it permits theoretical investigations of active suspensions via the fields. Nevertheless, these models require a high density present in the system to give the distribution $\Psi$ a tangible relevance compared to experiments.  In our case, we are studying a low-density suspension, as bacteria occupy a small fraction of the ROI. Due to this, we will consider an agent-based model even if that means there is less possibility to perform theoretical calculations.

\subsection{Overdamped dynamics}

We will consider cells with spherical shapes as a simplification of the model. In reality, cell shape depends on the growth phase of the bacteria culture, but \textit{E.coli}'s body is a spherocylinder typically with a \SI{0.5}{\micro\meter} radius and \SI{2}{\micro\meter} length. To describe the dynamics of single cells, we invoke Newton's second law, so we write: 

\begin{equation}
	m\ddot{\textbf{r}} = \sum \textbf{F} = \textbf{F}_h + \textbf{F}_f + \textbf{F}_b + \textbf{F}_c ,
\end{equation}

where we define four different force sources, \textbf{F}$_h$ the hydrodynamic friction produced by the fluid, \textbf{F}$_f$ the force produced by the flagella, \textbf{F}$_b$ the Brownian force due to collisions with the liquid molecules, and \textbf{F}$_c$ produced due to cell-cell and cell-surface collisions. We can deduct an important fact widely accepted in biophysics with the following analysis. Let's imagine we have a free of collisions \textit{E.coli} bacterium that dies and stops swimming, so $\textbf{F}_f = \textbf{F}_c =0$. The influence of $\textbf{F}_b$, in this case, is essential. Still, it will not be considered as its effect is stochastic. If we repeat the situation many times and average the ensemble, its contribution is expected to be zero. Then we are left with the equation:

\begin{equation}
	m\ddot{\textbf{r}} = \textbf{F}_h = -\gamma \dot{\textbf{r}} = - 6 \pi R \eta \dot{\textbf{r}},
\end{equation}

where we used Stokes' law to calculate the drag force coefficient $\gamma$ with $R=$\SI{0.5}{\micro\meter} being the particle radius and $\eta=$\SI{d-3}{\pascal\cdot\second} the viscosity. Since the mass of an \textit{E.coli} is $m=$\SI{d-12}{\gram}, the characteristic time of slowing down is given by $ \frac{m}{6 \pi R \eta} \approx $\SI{d-7}{\second}. This means that in the time scale of \SI{1}{\micro\second} the bacteria should have stopped. We record at 10 fps in our experiments, so the slow down is immediate for our time resolution. Then we are working on the overdamped limit, where inertia is meaningless, and it is customary to simplify the dynamics as:

\begin{equation} \label{eq:overdamped_model}
	\gamma \dot{\textbf{r}} = \textbf{F}_f + \textbf{F}_b + \textbf{F}_c.
\end{equation}

\subsection{Rotational diffusion}

As the effect of the fluid, we have considered two different effects. First $\textbf{F}_h$ a phenomenological drag force that represents the average effect of the liquid and second, $\textbf{F}_b$ of stochastic class accounting for the thermal fluctuations. The latter produces a mean square displacement of $\langle r^2 \rangle (t) = 4 D t $ where $D$ is the diffusion constant given by the fluctuation-dissipation theorem \cite{Soto2016KineticPhenomena}:

\begin{equation}
	D = \frac{k_bT}{\gamma} \approx \SI[per-mode = symbol]{1.5d-1}{\square\micro\meter \per \second},
\end{equation}
 
where the same value for $\gamma$ was used. This diffusion constant is purely translational. Nevertheless, the forces involved in cell-molecules collisions also exert torques on the bacteria, meaning that the orientation of swimming is subject to an equivalent stochastic process. This process has a rotational diffusion constant $D_r$ independent of $D$. For smooth-swimming \textit{E.coli} it has been measured to be $D_r=$\SI[per-mode = symbol]{0.057}{\square\radian \per \second} \cite{Drescher2011FluidScattering}. We can perceive the importance of rotational diffusion by focusing on the flagellar force of equation \eqref{eq:overdamped_model} as described in \cite{Lauga2020TheMotility}.

\begin{equation} \label{eq:flagellar force}
	 \dot{\textbf{r}} =\frac{\textbf{F}_f}{\gamma} = u \textbf{p},
\end{equation}

where $u=$\SI[per-mode = symbol]{20}{\micro\meter \per \second} is the mean speed of bacteria and $\textbf{p}$ the previously defined vector of orientation. We are ignoring collisions namely $\textbf{F}_c$ and exploiting the fact that the diffusion of $\textbf{F}_b$ is independent of the one we are currently descrbing. Integrating equation \eqref{eq:flagellar force} in time we obtain:

\begin{equation} \label{eq:positon}
	\textbf{r}(t) = \int_0^t u \textbf{p}(t^\prime)dt^\prime,
\end{equation}

Taking the dot product of \eqref{eq:flagellar force} and \eqref{eq:positon} we obtain an equation for the time derivative of the square displacement.

\begin{equation} 
	\dot{\textbf{r}} \cdot \textbf{r} = \frac{1}{2}\frac{d}{dt} (r^2) = u^2 \int_0^t  \textbf{p}(t^\prime) \cdot \textbf{p}(t)dt^\prime.
\end{equation}

The quantity $\textbf{p}(t^\prime) \cdot \textbf{p}(t)$ is the time correlation of the vector director \textbf{p} and varies from cell to cell, as the rotational diffusion is also stochastic. Nevertheless its mean for all particles dependes on the rotational diffusion coefficient as $e^{-2D_r(t-t^\prime)}$ \cite{Lauga2020TheMotility}. Therefore, averaging on the ensemble and integrating over both $t$ and $t^\prime$ we obtain:


\begin{align} 
    \frac{d}{dt} \langle r^2 \rangle &= 2u^2 \int_0^t e^{-2D_r(t-t^\prime)}dt^\prime, \\
    \langle r^2 \rangle (t) &= \frac{u^2}{D_r}\left( t + \frac{e^{-2D_rt}}{2D_r} - \frac{1}{2D_r} \right)
\end{align}

At short times,  $t << D_r^{-1} =$\SI{18}{\second} the exponential can be expanded in a taylor series $e^{-2D_rt}\approx 1 - 2D_rt + 2(D_r t^2) + \mathcal{O}(t^3)$ giving $\langle r^2 \rangle (t)  \approx (ut)^2$, which means that at short times particles swim is straight lines. This effect is more important than diffusion at this time scale as $4Dt$ is some order of magnitude lower. In the other case $t >> D_r^{-1}$ the mean square displacement is $\frac{u^2t}{D_r}$. Adding the translational diffusion discussed previously, we deduct an effective diffusion constant for long times $D_{eff}$ given by:

\begin{equation}
    D_{eff} = D + \frac{u^2}{4D_r}
\end{equation}

For the typical values mentioned, $D_{eff}\approx$\SI[per-mode = symbol]{2d3}{\square\micro\meter \per \second} which is four orders of magnitude greater than $D$. This means that bacteria swimming makes translational Brownian motion irrelevant compared to the effects of rotational diffusion. We conclude that $\textbf{F}_b$ can be set to zero for simplicity without losing relevant dynamics. We are left with the equation:

\begin{equation} \label{eq:final_model}
    \dot{\textbf{r}} = u\textbf{p} + \frac{1}{\gamma}\textbf{F}_c = u\textbf{p} + \tilde{\textbf{F}}_c  .
\end{equation}

\subsection{Surface effects}

\begin{wrapfigure}{r}{0.5\linewidth}
\centering
\includesvg[width=\linewidth]{imagenes/circular_trajectories}
\caption[Circular trajectories near walls]{Example of three circular trajectories seen in an experiment. The star symbol indicates the beginning of a track while the triangle the end. The video used for this figure had 12 trajectories with circular sections of 302, which is $4\%$. The three trajectories shown are the longest ones.}
\label{circular trajectories}
\end{wrapfigure}

In the introduction in chapter 1, we discussed multiple effects of the surface in bacteria. These effects depend on the swimming properties, body and flagella shape, and surface properties. Even so, we observe some of them in our experiments. Here we discuss which were considered relevant for the model.

First, we see circular swimming trajectories on the frontal surface, caused by hydrodynamic interactions between the cell and the boundary \cite{Lauga2006SwimmingBoundaries}. In figure \ref{circular trajectories} we display three example trajectories. The percentage of trajectories observed that display circular movement is $4\%$ for the experiment used for the figure. Thus, the phenomenon is marginal and so not implemented in the model. 

Second, we observe that cells rarely leave the focal plane. This behavior is produced due to hydrodynamic interactions that trap bacteria in the surface \cite{Sipos2015HydrodynamicWalls, Li2011AccumulationSurface}. This effect validates the 2-dimensional aspect of the model. 

Third and finally, we observe that bacteria interacting on the edges of the experiment, i.e. the curved and flat walls, suffer a torque that aligns them with the wall \cite{Bianchi2017HolographicBacteria}. This alignment has its origin on steric forces. To model this effect we define the vector $\textbf{p}$ via and angle $\theta$ and write:

\begin{align}
    \textbf{p} &= \cos{\theta}\hat{\textbf{x}}+\sin{\theta}\hat{\textbf{y}}, \\
    \dot{\theta} &= K (\textbf{p} \cdot \hat{\textbf{t}}_w)  (\textbf{p} \cdot \hat{\textbf{n}}_w) \Gamma(\textbf{r}, \textbf{p}),
    \label{eq:wall alignment}
\end{align}

where $\hat{\textbf{t}}_w$ and $\hat{\textbf{n}}_w$ are the tangential and normal vector of the wall in the closest point to the cell. These vectors satisfy $\hat{\textbf{n}}_w = \hat{\textbf{z}} \times \hat{\textbf{t}}_w$ and $\hat{\textbf{n}}_w$ points away from the wall. Also, $\Gamma(\textbf{r}, \textbf{p})$ is a step function equal to $1$ when the cell is in contact with the wall and $0$ otherwise. Contact is defined as the center of the particle $\textbf{r}$ is less than one radius near the closest point of the wall, and the direction of swimming suffices $\textbf{p} \cdot \textbf{n} < 0$ indicating the cell is going into the wall and not away from it. Equation \eqref{eq:wall alignment} will align the vector $\textbf{p}$ with the $\pm\hat{\textbf{t}}_w$ depending on the sign of $\textbf{p} \cdot \hat{\textbf{t}}_w$. This effect is only considered for the flat and curved wall as the frontal wall is already taken into account because simulations are 2-dimensional.

\section{Simulations}
 
\subsection{Final details of the model}

Equations \eqref{eq:final_model}-\eqref{eq:wall alignment} summarize the physics involved in the description of the system. We are only missing two aspects; rotational diffusion and the details of $\textbf{F}_c$. Rotational diffusion is naturally added to \eqref{eq:wall alignment} as a gaussian white noise $\eta$ \cite{Digregorio2018FullSeparation,Caporusso2020Motility-InducedSystem} meaning it has zero mean and its completly uncorrelated, $\langle \eta(t)\eta(t^\prime)  \rangle &= \delta (t-t^\prime)$. Meanwhile $\textbf{F}_c$ is considered as an elastic force for both cell-cell and cell-wall collisions. The final set of equations that are used in the model are:

\begin{align}
    \label{eq:dynamics of position}
    \dot{\textbf{r}}_i &= u\textbf{p}_i + \tilde{\textbf{F}}_c, \\
    \textbf{p}_i &= \cos{\theta_i}\hat{\textbf{x}}+\sin{\theta_i}\hat{\textbf{y}}, \\
    \label{eq:dynamics of angle}
    \dot{\theta}_i &= K (\textbf{p}_i \cdot \hat{\textbf{t}}_w)  (\textbf{p}_i \cdot \hat{\textbf{n}}_w) \Gamma(\textbf{r}_i, \textbf{p}_i) + \sqrt{2D_r}\eta(t), \\
    \label{eq:elastic force}
    \tilde{\textbf{F}}_c &=   k_{wall} (R-d_{iw}) \hat{\textbf{d}}_{iw}\Theta(R-d_{iw}) + \sum_{j \in \mathbb{NN}} k_{cell} (2R-d_{ij}) \hat{\textbf{d}}_{ij}\Theta(2R-d_{ij}).
\end{align}

All of the quantities used in these equations and on the model are described in table \ref{table:model parameters}. $\Theta$ is the regular Heaviside function. These equations are written for the dynamics of a circular particle $i$. But involve other entities such as the wall and the nearest neighbors. Equations \eqref{eq:dynamics of position}-\eqref{eq:elastic force} are called Langevin equations due to their stochastic nature.

% Tabla del modelo
\begin{table}[!h]
   \centering
    \small
    \caption[Summary of the quantities used in the simulations]{All of the quantities used for the model, with their description and values. For variables the value column will be empty and for parameters that take a range of values, that column will have a $-$ symbol and in chapter 4, results will have the value specified. }
    \begin{tabularx}{\textwidth}{lXl}
    \hline\noalign{\smallskip}
         Variable  & Description & \quad   \\
    \noalign{\smallskip}\hline\noalign{\smallskip}
         \textbf{r} & Position of the particle $i$. & \quad \\ 
         \textbf{p} & Direction of swimming of the particle $i$, defined by the angle $\theta_i$. & \quad \\
         $\eta$ & White noise associated with rotational diffusion. & \quad \\
         $\hat{\textbf{t}}_w$ & Unitary vector tangent to the wall in the closest point to the cell $i$. & \quad \\
         $\hat{\textbf{n}}_{w}$ & Unitary vector normal to the wall in the closest point to the cell $i$. The normal vectors points away from the wall. & \quad \\
         $\hat{\textbf{d}}_{iw}$ & Unitary vector pointing from the closest point in the wall to the cell $i$. & \quad \\
         $d_{iw}$ & Distance between the cell $i$ and the closest point of the wall. & \quad \\
         $\mathbb{NN}$ & Set of nearest-neighbors of the particle $i$. & \quad \\
         $\hat{\textbf{d}}_{ij}$ & Unitary vector pointing from the neighbor $j$ to the cell $i$. & \quad \\
         $d_{ij}$ & Distance between the cell $i$ and the neighbor $j$. & \quad \\
    \hline\noalign{\smallskip}
        Parameter  & Description & Value   \\
        $\rho$ & Density of cells. This density relates to number of particles seen in the focal plane. & \SI[per-mode = symbol]{3d-3}{\cells \per \square\micro\meter} \\
        $u$ & Speed of swimming. & \SI[per-mode = symbol]{20}{\micro\meter\per\second} \\
        $R$ & Radius of the cell. & \SI{0.5}{\micro\meter} \\
         $K$ & Magnitude of the aligment with the wall. & - \\ 
         $D_r$ & Rotational diffusion coefficient. & - \\ 
         $k_{wall}$ & Elastic constant for cell-wall collisions. & \SI[per-mode = symbol]{1d9}{\newton\per\meter} \\ 
         $k_{cell}$ & Elastic constant for cell-cell collisions. & - \\
         $\Delta t$ & Time step for the integration of the equations. & \SI{d-3}{\second} \\
         $\Delta t_r$ & Time step for the recording of data. & \SI{d-1}{\second} \\
         $T$ & Time duration of the simulations. & \SI{200}{\second} \\
    \hline\noalign{\smallskip}
    \end{tabularx}
    \label{table:model parameters}
\end{table}

Quantities with the lower index $w$ depend on the wall. Simulations have a flat and a curved wall with amplitude $A$ and wavelength $\lambda$. The parametric definition of the walls positions are $y=y_f$ for the flat and $y=y_c+A\sin{\frac{2\pi x}{\lambda}}$ for the curved wall. The values of $y_f, y_c$ are so that the mean distance between walls is \SI{100}{\micro\meter}. Also, periodic boundary conditions will be applied, so to avoid problems with discontinuities in the wall, the x-axis length is scaled to have an exact number of wavelengths and be greater than \SI{300}{\micro\meter}. The total particle number is adjusted so that all simulations have the same density of particles $\rho$. These dimensions are similar to the experiments.

\subsection{Numerical integration}

Typically, there are many options to solve a differential equation numerically. In this case, the equations are somewhat simple, so it is tempting to integrate \eqref{eq:dynamics of position} and \eqref{eq:dynamics of angle} with Euler's method. That integration method is a first-order method which means that during a step $\Delta t$ of integration, the integrands are taken as constants. Therefore we would obtain:

\begin{align}
    \textbf{r}_i(t+\Delta t) &=  \textbf{r}_i(t) + \Delta t(u \textbf{p}_i(t) + \tilde{\textbf{F}}_c(t)), \\
    \label{eq:wrong integration}
    \theta_i(t+\Delta t) &=  \theta_i(t) + \Delta t(K (\textbf{p}_i(t) \cdot \hat{\textbf{t}}_w)  (\textbf{p}_i(t) \cdot \hat{\textbf{n}}_w) \Gamma(\textbf{r}_i(t), \textbf{p}_i(t)) + \sqrt{2D_r}\eta(t)).
\end{align}

Everything looks good, but there is a problem with the last term of equation \eqref{eq:wrong integration}. The problem relies on the timescale in which $\eta$ acts. Remember that $\eta$ represents the torque produced by collisions with the liquid molecules. These thermal fluctuations have a characteristic time between collisions of $\sim$\SI{1.9d-13}{\second} \cite{Soto2016KineticPhenomena}. That means we have two options, either we consider a minimal time step $\Delta t$ so the hypothesis that $\eta$ is constant in the interval is true, or we treat $\eta$ separately. The latter is the best option, as $\eta$'s timescale is much lower than all the others present in the system. The correct equations are:

\begin{align}
    \textbf{r}_i(t+\Delta t) &=  \textbf{r}_i(t) + \Delta t(u \textbf{p}_i(t) + \tilde{\textbf{F}}_c(t)), \\
    \theta_i(t+\Delta t) &=  \theta_i(t) + \Delta t(K (\textbf{p}_i(t) \cdot \hat{\textbf{t}}_w)  (\textbf{p}_i(t) \cdot \hat{\textbf{n}}_w) \Gamma(\textbf{r}_i(t), \textbf{p}_i(t))) \nonumber \\
     &\ \ \ + \sqrt{2D_r} \int_t^{t+\Delta t}\eta(t^\prime)dt^\prime .
\end{align}


Then a new problem arises, how do we calculate the integral of $\eta$. We can find the answer on stochastic calculus, and here we describe one of the possible demonstrations. First, we treat $\eta$ as what it is, a discrete function representing all the collisions with liquid molecules.

\begin{equation}
    \int_t^{t+\Delta t}\eta(t^\prime)dt^\prime = \sum_{n=1}^N \eta_n,
\end{equation}

where $\eta_n$ is proportional to the angle displacement produced in a certain collision $n$. The magnitude of these rotations is contained in $D_r$ so that we can treat them as standard Gaussians. As we stated, these collision displacements are also uncorrelated. Therefore, we can use that the distribution of the sum of two gaussian uncorrelated variables, $x\sim\mathcal{N}(\mu_x,\sigma_x^2)$ and $y\sim\mathcal{N}(\mu_y,\sigma_y^2)$, is given by $\mathcal{N}(\mu_x+\mu_y,\sigma_x^2+\sigma_y^2)$ and by induction obtain:

\begin{equation}
    \sum_{n=1}^N \eta_n \sim \mathcal{N}(0,N) = \sqrt{N}\mathcal{N}(0,1),
\end{equation}

were the last equality is in the sense of probability density distribution. Then, the total number of collisions $N$ is proportional to $\Delta t$, for instance $N=\alpha \Delta t$, where $\alpha$ is a rate of collisions that in reality is contained on $D_r$ together with the magnitude of angle variations due to collisions. Then we can write:

\begin{align}
    \textbf{r}_i(t+\Delta t) &=  \textbf{r}_i(t) + \Delta t(u \textbf{p}_i(t) + \tilde{\textbf{F}}_c(t)), \\
    \label{eq:correct integration}
    \theta_i(t+\Delta t) &=  \theta_i(t) + \Delta t(K (\textbf{p}_i(t) \cdot \hat{\textbf{t}}_w)  (\textbf{p}_i(t) \cdot \hat{\textbf{n}}_w) \Gamma(\textbf{r}_i(t), \textbf{p}_i(t))) \nonumber \\
     &\ \ \ + \sqrt{2D_r\Delta t} \eta(t) .
\end{align}

The result of equation \eqref{eq:correct integration} is really important for the consistency of the numerical integration, since the equation $\langle \eta(t)\eta(t^\prime)  \rangle &= \delta (t-t^\prime)$ implies that $\eta$ has units of \SI{}{\second^{-1/2}} so multiplying by $\Delta t$ instead of $\sqrt{\Delta t}$ is dimensionally wrong.

\subsection{Algorithm}

The algorithm used in the simulations is described in the following steps.

\begin{itemize}
    \item[1.] Create a random initial condition for all the particles positions and swimming directions. Then start iterating the time steps.
    \item[2.] Update the closest point of the wall for all particles close to the walls.
    \item[3.] Integrate the equation for $\theta_i$.
    \item[4.] Determine the $\mathbb{NN}$ set for all particles.
    \item[5.] Calculate the force $\tilde{\textbf{F}}_c$.
    \item[6.] Integrate the equation for $\textbf{r}_i$.
    \item[7.] Every time interval $\Delta t_r$ record the relevant data. This avoids high time correlations between measurements.
    \item[8.] Stop when a time $T$ has passed.
\end{itemize}

For the step number 2, we calculate the closest point in the walls to the cell by dividing the interval $[x_i-R, x_i +R]$ in 300 points where $x_i = \textbf{r}_i \cdot \hat{x}$. Then, we calculate the distance between the position obtained with the parametric definition of the wall and the cell for each point. The point with the lowest distance is the closest with a precision of \SI{3d-3}{\micro\meter}. This "brute force" algorithm works because a point outside of the interval $[x_i-R, x_i +R]$ is not in contact with the cell and is convenient because the distance to the wall as a function of the x-axis has many local minimums. The algorithm was implemented in a c++ program with object oriented programming.\ Results are left for chapter 4.

\subsection{Intensity profiles in simulations}

\textcolor{red}{Note: Add frames of simulations or trajectories}